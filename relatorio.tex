% Options for packages loaded elsewhere
\PassOptionsToPackage{unicode}{hyperref}
\PassOptionsToPackage{hyphens}{url}
%
\documentclass[
]{article}
\usepackage{amsmath,amssymb}
\usepackage{lmodern}
\usepackage{iftex}
\ifPDFTeX
  \usepackage[T1]{fontenc}
  \usepackage[utf8]{inputenc}
  \usepackage{textcomp} % provide euro and other symbols
\else % if luatex or xetex
  \usepackage{unicode-math}
  \defaultfontfeatures{Scale=MatchLowercase}
  \defaultfontfeatures[\rmfamily]{Ligatures=TeX,Scale=1}
\fi
% Use upquote if available, for straight quotes in verbatim environments
\IfFileExists{upquote.sty}{\usepackage{upquote}}{}
\IfFileExists{microtype.sty}{% use microtype if available
  \usepackage[]{microtype}
  \UseMicrotypeSet[protrusion]{basicmath} % disable protrusion for tt fonts
}{}
\makeatletter
\@ifundefined{KOMAClassName}{% if non-KOMA class
  \IfFileExists{parskip.sty}{%
    \usepackage{parskip}
  }{% else
    \setlength{\parindent}{0pt}
    \setlength{\parskip}{6pt plus 2pt minus 1pt}}
}{% if KOMA class
  \KOMAoptions{parskip=half}}
\makeatother
\usepackage{xcolor}
\IfFileExists{xurl.sty}{\usepackage{xurl}}{} % add URL line breaks if available
\IfFileExists{bookmark.sty}{\usepackage{bookmark}}{\usepackage{hyperref}}
\hypersetup{
  pdftitle={Diferenças na Área de Dados, no Mundo Corporativo, por Gênero},
  hidelinks,
  pdfcreator={LaTeX via pandoc}}
\urlstyle{same} % disable monospaced font for URLs
\usepackage[margin=1in]{geometry}
\usepackage{color}
\usepackage{fancyvrb}
\newcommand{\VerbBar}{|}
\newcommand{\VERB}{\Verb[commandchars=\\\{\}]}
\DefineVerbatimEnvironment{Highlighting}{Verbatim}{commandchars=\\\{\}}
% Add ',fontsize=\small' for more characters per line
\usepackage{framed}
\definecolor{shadecolor}{RGB}{248,248,248}
\newenvironment{Shaded}{\begin{snugshade}}{\end{snugshade}}
\newcommand{\AlertTok}[1]{\textcolor[rgb]{0.94,0.16,0.16}{#1}}
\newcommand{\AnnotationTok}[1]{\textcolor[rgb]{0.56,0.35,0.01}{\textbf{\textit{#1}}}}
\newcommand{\AttributeTok}[1]{\textcolor[rgb]{0.77,0.63,0.00}{#1}}
\newcommand{\BaseNTok}[1]{\textcolor[rgb]{0.00,0.00,0.81}{#1}}
\newcommand{\BuiltInTok}[1]{#1}
\newcommand{\CharTok}[1]{\textcolor[rgb]{0.31,0.60,0.02}{#1}}
\newcommand{\CommentTok}[1]{\textcolor[rgb]{0.56,0.35,0.01}{\textit{#1}}}
\newcommand{\CommentVarTok}[1]{\textcolor[rgb]{0.56,0.35,0.01}{\textbf{\textit{#1}}}}
\newcommand{\ConstantTok}[1]{\textcolor[rgb]{0.00,0.00,0.00}{#1}}
\newcommand{\ControlFlowTok}[1]{\textcolor[rgb]{0.13,0.29,0.53}{\textbf{#1}}}
\newcommand{\DataTypeTok}[1]{\textcolor[rgb]{0.13,0.29,0.53}{#1}}
\newcommand{\DecValTok}[1]{\textcolor[rgb]{0.00,0.00,0.81}{#1}}
\newcommand{\DocumentationTok}[1]{\textcolor[rgb]{0.56,0.35,0.01}{\textbf{\textit{#1}}}}
\newcommand{\ErrorTok}[1]{\textcolor[rgb]{0.64,0.00,0.00}{\textbf{#1}}}
\newcommand{\ExtensionTok}[1]{#1}
\newcommand{\FloatTok}[1]{\textcolor[rgb]{0.00,0.00,0.81}{#1}}
\newcommand{\FunctionTok}[1]{\textcolor[rgb]{0.00,0.00,0.00}{#1}}
\newcommand{\ImportTok}[1]{#1}
\newcommand{\InformationTok}[1]{\textcolor[rgb]{0.56,0.35,0.01}{\textbf{\textit{#1}}}}
\newcommand{\KeywordTok}[1]{\textcolor[rgb]{0.13,0.29,0.53}{\textbf{#1}}}
\newcommand{\NormalTok}[1]{#1}
\newcommand{\OperatorTok}[1]{\textcolor[rgb]{0.81,0.36,0.00}{\textbf{#1}}}
\newcommand{\OtherTok}[1]{\textcolor[rgb]{0.56,0.35,0.01}{#1}}
\newcommand{\PreprocessorTok}[1]{\textcolor[rgb]{0.56,0.35,0.01}{\textit{#1}}}
\newcommand{\RegionMarkerTok}[1]{#1}
\newcommand{\SpecialCharTok}[1]{\textcolor[rgb]{0.00,0.00,0.00}{#1}}
\newcommand{\SpecialStringTok}[1]{\textcolor[rgb]{0.31,0.60,0.02}{#1}}
\newcommand{\StringTok}[1]{\textcolor[rgb]{0.31,0.60,0.02}{#1}}
\newcommand{\VariableTok}[1]{\textcolor[rgb]{0.00,0.00,0.00}{#1}}
\newcommand{\VerbatimStringTok}[1]{\textcolor[rgb]{0.31,0.60,0.02}{#1}}
\newcommand{\WarningTok}[1]{\textcolor[rgb]{0.56,0.35,0.01}{\textbf{\textit{#1}}}}
\usepackage{graphicx}
\makeatletter
\def\maxwidth{\ifdim\Gin@nat@width>\linewidth\linewidth\else\Gin@nat@width\fi}
\def\maxheight{\ifdim\Gin@nat@height>\textheight\textheight\else\Gin@nat@height\fi}
\makeatother
% Scale images if necessary, so that they will not overflow the page
% margins by default, and it is still possible to overwrite the defaults
% using explicit options in \includegraphics[width, height, ...]{}
\setkeys{Gin}{width=\maxwidth,height=\maxheight,keepaspectratio}
% Set default figure placement to htbp
\makeatletter
\def\fps@figure{htbp}
\makeatother
\setlength{\emergencystretch}{3em} % prevent overfull lines
\providecommand{\tightlist}{%
  \setlength{\itemsep}{0pt}\setlength{\parskip}{0pt}}
\setcounter{secnumdepth}{-\maxdimen} % remove section numbering
\ifLuaTeX
  \usepackage{selnolig}  % disable illegal ligatures
\fi

\title{Diferenças na Área de Dados, no Mundo Corporativo, por Gênero}
\author{}
\date{\vspace{-2.5em}2022-07-02}

\begin{document}
\maketitle

Não é novidade para ninguém que vivemos em uma sociedade que, há
séculos, tende a favorecer os homens em relação às mulheres em diversos
aspectos e no meio profissional não é diferente. Embora isto venha
mudando paulatinamente, ainda é possível constatar muitos problemas
oriundos da diferença de gênero. Por isso, estudos sobre o tópico se
fazem necessários se quisermos superar o problema em questão na
sociedade e o ``State of Data Brazil 2021'', maior pesquisa sobre a área
de dados no Brasil, permite uma interessante investigação sobre o tema
no que é hoje uma das áreas mais modernas e aquecidas do mercado. Como a
diferença entre os gêneros está presente a área de dados, que sofreu
profundas transformações nos últimos 20 anos e vem se mostrando cada vez
mais como um dos campos que melhor oferecem oportunidades.

Primeiro, é importante pontuar que, para as análises mostradas a seguir,
foram consideradas apenas pessoas cuja situação de trabalho era
``Empreendedor ou Empregado (CNPJ)'' e ``Empregado (CLT)'', pois o
objetivo é investigar as possíveis diferenças de gênero no mundo
corporativo, portanto, pessoas que trabalham como freelancer,
acadêmicos, servidor público foram excluídas, assim como estagiários
(somente pessoas devidamente empregadas foram inclusas). Também foram
deixadas de lado pessoas que se identificaram com a categoria de gênero
``Outros'' (o objetivo é investigar possiveis diferenças entre homens e
mulheres e, além disso, apenas 5 pessoas entraram nesta categoria, o que
complicaria comparações com os outros gêneros).

\begin{Shaded}
\begin{Highlighting}[]
\DocumentationTok{\#\#\#Importando a base de dados\#\#\#\#\#\#\#\#\#\#\#\#\#\#\#\#\#\#\#\#\#\#\#\#\#\#}

\NormalTok{SoD2021}\OtherTok{\textless{}{-}}\NormalTok{readr}\SpecialCharTok{::}\FunctionTok{read\_csv}\NormalTok{(}\StringTok{\textquotesingle{}data/State of Data 2021 {-} Dataset {-} Pgina1.csv\textquotesingle{}}\NormalTok{)}
\StringTok{\textasciigrave{}}\AttributeTok{\%\textgreater{}\%}\StringTok{\textasciigrave{}}\OtherTok{=}\NormalTok{magrittr}\SpecialCharTok{::}\StringTok{\textasciigrave{}}\AttributeTok{\%\textgreater{}\%}\StringTok{\textasciigrave{}}
\StringTok{\textasciigrave{}}\AttributeTok{\%notin\%}\StringTok{\textasciigrave{}}\OtherTok{=}\FunctionTok{Negate}\NormalTok{(}\StringTok{\textasciigrave{}}\AttributeTok{\%in\%}\StringTok{\textasciigrave{}}\NormalTok{)}

\CommentTok{\#Preparando a base para as análises}
\CommentTok{\#Incluo na base apenas as categorias "Empreendedor ou Empregado (CNPJ)" e}
\CommentTok{\#"Empregado (CLT)" da variável "Qual sua situação atual de trabalho?".}
\CommentTok{\#Excluo a categoria de gênero "Outros".}
\CommentTok{\#Além disso, reordeno as categorias das variáveis "nível de ensino",}
\CommentTok{\#"faixa salarial", "experiência na área de dados" e "experiência na área de TI"}

\NormalTok{df}\OtherTok{=}\NormalTok{SoD2021}\SpecialCharTok{\%\textgreater{}\%}\NormalTok{dplyr}\SpecialCharTok{::}\FunctionTok{filter}\NormalTok{(}
  \StringTok{\textasciigrave{}}\AttributeTok{(\textquotesingle{}P2\_a \textquotesingle{}, \textquotesingle{}Qual sua situação atual de trabalho?\textquotesingle{})}\StringTok{\textasciigrave{}}\SpecialCharTok{\%in\%}\FunctionTok{c}\NormalTok{(}
    \StringTok{\textquotesingle{}Empreendedor ou Empregado (CNPJ)\textquotesingle{}}\NormalTok{, }\StringTok{\textquotesingle{}Empregado (CLT)\textquotesingle{}}\NormalTok{) }\SpecialCharTok{\&}
    \StringTok{\textasciigrave{}}\AttributeTok{(\textquotesingle{}P1\_b \textquotesingle{}, \textquotesingle{}Genero\textquotesingle{})}\StringTok{\textasciigrave{}}\SpecialCharTok{!=}\StringTok{\textquotesingle{}Outro\textquotesingle{}}\NormalTok{ )}\SpecialCharTok{\%\textgreater{}\%}
\NormalTok{  dplyr}\SpecialCharTok{::}\FunctionTok{mutate}\NormalTok{(}\StringTok{\textasciigrave{}}\AttributeTok{(\textquotesingle{}P2\_h \textquotesingle{}, \textquotesingle{}Faixa salarial\textquotesingle{})}\StringTok{\textasciigrave{}}\OtherTok{=}\NormalTok{ forcats}\SpecialCharTok{::}\FunctionTok{lvls\_reorder}\NormalTok{(}
  \FunctionTok{as.factor}\NormalTok{(}\StringTok{\textasciigrave{}}\AttributeTok{(\textquotesingle{}P2\_h \textquotesingle{}, \textquotesingle{}Faixa salarial\textquotesingle{})}\StringTok{\textasciigrave{}}\NormalTok{),}
  \FunctionTok{c}\NormalTok{(}\DecValTok{13}\NormalTok{, }\DecValTok{2}\NormalTok{, }\DecValTok{5}\NormalTok{, }\DecValTok{8}\NormalTok{, }\DecValTok{10}\NormalTok{, }\DecValTok{11}\NormalTok{, }\DecValTok{12}\NormalTok{, }\DecValTok{3}\NormalTok{, }\DecValTok{4}\NormalTok{, }\DecValTok{6}\NormalTok{, }\DecValTok{7}\NormalTok{, }\DecValTok{9}\NormalTok{, }\DecValTok{1}\NormalTok{)),}
  \StringTok{\textasciigrave{}}\AttributeTok{(\textquotesingle{}P1\_h \textquotesingle{}, \textquotesingle{}Nivel de Ensino\textquotesingle{})}\StringTok{\textasciigrave{}}\OtherTok{=}\NormalTok{ forcats}\SpecialCharTok{::}\FunctionTok{lvls\_reorder}\NormalTok{(}
    \FunctionTok{as.factor}\NormalTok{(}\StringTok{\textasciigrave{}}\AttributeTok{(\textquotesingle{}P1\_h \textquotesingle{}, \textquotesingle{}Nivel de Ensino\textquotesingle{})}\StringTok{\textasciigrave{}}\NormalTok{), }\FunctionTok{c}\NormalTok{(}\DecValTok{5}\NormalTok{, }\DecValTok{2}\NormalTok{, }\DecValTok{3}\NormalTok{, }\DecValTok{6}\NormalTok{, }\DecValTok{4}\NormalTok{ , }\DecValTok{1}\NormalTok{, }\DecValTok{7}\NormalTok{)),}
  \StringTok{\textasciigrave{}}\AttributeTok{(\textquotesingle{}P2\_i \textquotesingle{}, \textquotesingle{}Quanto tempo de experiência na área de dados você tem?\textquotesingle{})}\StringTok{\textasciigrave{}}\OtherTok{=}\NormalTok{ forcats}\SpecialCharTok{::}\FunctionTok{lvls\_reorder}\NormalTok{(}
    \FunctionTok{as.factor}\NormalTok{(}\StringTok{\textasciigrave{}}\AttributeTok{(\textquotesingle{}P2\_i \textquotesingle{}, \textquotesingle{}Quanto tempo de experiência na área de dados você tem?\textquotesingle{})}\StringTok{\textasciigrave{}}\NormalTok{),}
    \FunctionTok{c}\NormalTok{(}\DecValTok{7}\NormalTok{, }\DecValTok{6}\NormalTok{, }\DecValTok{1}\NormalTok{, }\DecValTok{2}\NormalTok{, }\DecValTok{3}\NormalTok{, }\DecValTok{4}\NormalTok{, }\DecValTok{5}\NormalTok{)),}
  \StringTok{\textasciigrave{}}\AttributeTok{(\textquotesingle{}P2\_j \textquotesingle{}, \textquotesingle{}Quanto tempo de experiência na área de TI/Engenharia de Software você teve antes de começar a trabalhar na área de dados?\textquotesingle{})}\StringTok{\textasciigrave{}}
  \OtherTok{=}\NormalTok{ forcats}\SpecialCharTok{::}\FunctionTok{lvls\_reorder}\NormalTok{(}
    \FunctionTok{as.factor}\NormalTok{(}\StringTok{\textasciigrave{}}\AttributeTok{(\textquotesingle{}P2\_j \textquotesingle{}, \textquotesingle{}Quanto tempo de experiência na área de TI/Engenharia de Software você teve antes de começar a trabalhar na área de dados?\textquotesingle{})}\StringTok{\textasciigrave{}}
\NormalTok{),}
    \FunctionTok{c}\NormalTok{(}\DecValTok{7}\NormalTok{, }\DecValTok{6}\NormalTok{, }\DecValTok{1}\NormalTok{, }\DecValTok{2}\NormalTok{, }\DecValTok{3}\NormalTok{, }\DecValTok{4}\NormalTok{, }\DecValTok{5}\NormalTok{))}
\NormalTok{)}

\CommentTok{\#Substituindo os valores "NA"s da variável "Nível do cargo", pelo valor "Gestor"}

\NormalTok{df}\SpecialCharTok{$}\StringTok{\textasciigrave{}}\AttributeTok{(\textquotesingle{}P2\_g \textquotesingle{}, \textquotesingle{}Nivel\textquotesingle{})}\StringTok{\textasciigrave{}}\OtherTok{=}\FunctionTok{replace}\NormalTok{(df}\SpecialCharTok{$}\StringTok{\textasciigrave{}}\AttributeTok{(\textquotesingle{}P2\_g \textquotesingle{}, \textquotesingle{}Nivel\textquotesingle{})}\StringTok{\textasciigrave{}}\NormalTok{,}
        \FunctionTok{is.na}\NormalTok{(df}\SpecialCharTok{$}\StringTok{\textasciigrave{}}\AttributeTok{(\textquotesingle{}P2\_g \textquotesingle{}, \textquotesingle{}Nivel\textquotesingle{})}\StringTok{\textasciigrave{}}\NormalTok{), }\StringTok{\textquotesingle{}Gestor\textquotesingle{}}\NormalTok{)}



\CommentTok{\#Reordenando as categorias da variável "Nível do cargo"}

\NormalTok{df}\OtherTok{=}\NormalTok{ df}\SpecialCharTok{\%\textgreater{}\%}\NormalTok{dplyr}\SpecialCharTok{::}\FunctionTok{mutate}\NormalTok{(}
  \StringTok{\textasciigrave{}}\AttributeTok{(\textquotesingle{}P2\_g \textquotesingle{}, \textquotesingle{}Nivel\textquotesingle{})}\StringTok{\textasciigrave{}}\OtherTok{=}\NormalTok{forcats}\SpecialCharTok{::}\FunctionTok{lvls\_reorder}\NormalTok{(}
  \FunctionTok{as.factor}\NormalTok{(}\StringTok{\textasciigrave{}}\AttributeTok{(\textquotesingle{}P2\_g \textquotesingle{}, \textquotesingle{}Nivel\textquotesingle{})}\StringTok{\textasciigrave{}}\NormalTok{),}
  \FunctionTok{c}\NormalTok{( }\DecValTok{2}\NormalTok{, }\DecValTok{3}\NormalTok{, }\DecValTok{4}\NormalTok{, }\DecValTok{1}\NormalTok{))}
\NormalTok{  )}

\CommentTok{\#Criando a categoria "Acima de R$ 16.001/mês" em "faixa salarial"}

\FunctionTok{levels}\NormalTok{(df}\SpecialCharTok{$}\StringTok{\textasciigrave{}}\AttributeTok{(\textquotesingle{}P2\_h \textquotesingle{}, \textquotesingle{}Faixa salarial\textquotesingle{})}\StringTok{\textasciigrave{}}\NormalTok{)[}
  \FunctionTok{levels}\NormalTok{(df}\SpecialCharTok{$}\StringTok{\textasciigrave{}}\AttributeTok{(\textquotesingle{}P2\_h \textquotesingle{}, \textquotesingle{}Faixa salarial\textquotesingle{})}\StringTok{\textasciigrave{}}\NormalTok{)}\SpecialCharTok{\%in\%}\FunctionTok{c}\NormalTok{(}\StringTok{"de R$ 16.001/mês a R$20.000/mês"}\NormalTok{,}\StringTok{"de R$ 20.001/mês a R$ 25.000/mês"}\NormalTok{,}
\StringTok{"de R$ 25.001/mês a R$ 30.000/mês"}\NormalTok{,}\StringTok{"de R$ 30.001/mês a R$ 40.000/mês"}\NormalTok{,}
\StringTok{"Acima de R$ 40.001/mês"}\NormalTok{)] }\OtherTok{\textless{}{-}} \StringTok{"Acima de R$ 16.001/mês"}


\DocumentationTok{\#\#Experiência\#\#\#\#\#\#\#\#\#\#\#\#\#}

\CommentTok{\#Alterando o nome das variáveis "Experiência na área de dados" e}
\CommentTok{\# "Experiência na área de TI" para simplificar a manipulação}

\FunctionTok{colnames}\NormalTok{(df)[}\FunctionTok{c}\NormalTok{(}\DecValTok{20}\NormalTok{,}\DecValTok{21}\NormalTok{)]}\OtherTok{=}\FunctionTok{c}\NormalTok{(}\StringTok{\textquotesingle{}Experiência na área de dados\textquotesingle{}}\NormalTok{,}
                                  \StringTok{\textquotesingle{}Experiência na área de TI\textquotesingle{}}\NormalTok{)}


\CommentTok{\#renomeando categorias das variáveis "Experiência na área de dados" e}
\CommentTok{\# "Experiência na área de TI"}

\NormalTok{df}\OtherTok{=}\NormalTok{df}\SpecialCharTok{\%\textgreater{}\%}\NormalTok{dplyr}\SpecialCharTok{::}\FunctionTok{mutate}\NormalTok{(}\StringTok{\textasciigrave{}}\AttributeTok{Experiência na área de dados}\StringTok{\textasciigrave{}}\OtherTok{=}\NormalTok{forcats}\SpecialCharTok{::}\FunctionTok{lvls\_revalue}\NormalTok{(}
  \StringTok{\textasciigrave{}}\AttributeTok{Experiência na área de dados}\StringTok{\textasciigrave{}}\NormalTok{, }\FunctionTok{c}\NormalTok{(}\StringTok{\textquotesingle{}Sem experiência\textquotesingle{}}\NormalTok{, }\StringTok{\textquotesingle{}Menos de 1 ano\textquotesingle{}}\NormalTok{,}
                                    \StringTok{\textquotesingle{}1 a 2 anos\textquotesingle{}}\NormalTok{, }\StringTok{\textquotesingle{}2 a 3 anos\textquotesingle{}}\NormalTok{, }\StringTok{\textquotesingle{}4 a 5 anos\textquotesingle{}}\NormalTok{,}
                                    \StringTok{\textquotesingle{}6 a 10 anos\textquotesingle{}}\NormalTok{, }\StringTok{\textquotesingle{}Mais de 10 anos\textquotesingle{}}\NormalTok{))}
\NormalTok{)}

\NormalTok{df}\OtherTok{=}\NormalTok{df}\SpecialCharTok{\%\textgreater{}\%}\NormalTok{dplyr}\SpecialCharTok{::}\FunctionTok{mutate}\NormalTok{(}\StringTok{\textasciigrave{}}\AttributeTok{Experiência na área de TI}\StringTok{\textasciigrave{}}\OtherTok{=}\NormalTok{forcats}\SpecialCharTok{::}\FunctionTok{lvls\_revalue}\NormalTok{(}
  \StringTok{\textasciigrave{}}\AttributeTok{Experiência na área de TI}\StringTok{\textasciigrave{}}\NormalTok{, }\FunctionTok{c}\NormalTok{(}\StringTok{\textquotesingle{}Sem experiência\textquotesingle{}}\NormalTok{, }\StringTok{\textquotesingle{}Menos de 1 ano\textquotesingle{}}\NormalTok{,}
                                    \StringTok{\textquotesingle{}1 a 2 anos\textquotesingle{}}\NormalTok{, }\StringTok{\textquotesingle{}2 a 3 anos\textquotesingle{}}\NormalTok{, }\StringTok{\textquotesingle{}4 a 5 anos\textquotesingle{}}\NormalTok{,}
                                    \StringTok{\textquotesingle{}6 a 10 anos\textquotesingle{}}\NormalTok{, }\StringTok{\textquotesingle{}Mais de 10 anos\textquotesingle{}}\NormalTok{))}
\NormalTok{)}
\end{Highlighting}
\end{Shaded}

Logo de cara, observamos que a área de dados é amplamente dominada pelos
homens, pois mais de 81\% dos entrevistados (levando em conta as
considerações feitas anteriormente) são homens e este número não é muito
diferente do da versão da mesma pesquisa realizada em 2019, na ocasião
chamada de ``Data Hackers Survey 2019'', onde quase 83\% dos
entrevistados (com as mesmas considerações) foram homens. Assim,
imaginar eventuais desfavorecimentos sofridos pelas mulheres em uma área
predominantemente masculina não é um absurdo.

\begin{Shaded}
\begin{Highlighting}[]
\CommentTok{\#Pessoas por gênero {-}State of Data 2021}

\NormalTok{df}\SpecialCharTok{\%\textgreater{}\%}\NormalTok{dplyr}\SpecialCharTok{::}\FunctionTok{count}\NormalTok{(}\StringTok{\textasciigrave{}}\AttributeTok{(\textquotesingle{}P1\_b \textquotesingle{}, \textquotesingle{}Genero\textquotesingle{})}\StringTok{\textasciigrave{}}\NormalTok{)}\SpecialCharTok{\%\textgreater{}\%}
\NormalTok{  dplyr}\SpecialCharTok{::}\FunctionTok{mutate}\NormalTok{(}\AttributeTok{perc =}\NormalTok{ n}\SpecialCharTok{/}\FunctionTok{sum}\NormalTok{(n)}\SpecialCharTok{*}\DecValTok{100}\NormalTok{)}\SpecialCharTok{\%\textgreater{}\%}\NormalTok{.[,}\SpecialCharTok{{-}}\DecValTok{2}\NormalTok{]}
\end{Highlighting}
\end{Shaded}

\begin{verbatim}
## # A tibble: 2 x 2
##   `('P1_b ', 'Genero')`  perc
##   <chr>                 <dbl>
## 1 Feminino               18.6
## 2 Masculino              81.4
\end{verbatim}

\begin{Shaded}
\begin{Highlighting}[]
\DocumentationTok{\#\#\#Importando a base de dados\#\#\#\#\#\#\#\#\#\#\#\#\#\#\#\#\#\#\#\#\#\#\#\#\#\#}

\NormalTok{SoD2019}\OtherTok{\textless{}{-}}\NormalTok{readr}\SpecialCharTok{::}\FunctionTok{read\_csv}\NormalTok{(}\StringTok{\textquotesingle{}data/datahackers{-}survey{-}2019{-}anonymous{-}responses.csv\textquotesingle{}}\NormalTok{)}
\StringTok{\textasciigrave{}}\AttributeTok{\%\textgreater{}\%}\StringTok{\textasciigrave{}}\OtherTok{=}\NormalTok{magrittr}\SpecialCharTok{::}\StringTok{\textasciigrave{}}\AttributeTok{\%\textgreater{}\%}\StringTok{\textasciigrave{}}
\StringTok{\textasciigrave{}}\AttributeTok{\%notin\%}\StringTok{\textasciigrave{}}\OtherTok{=}\FunctionTok{Negate}\NormalTok{(}\StringTok{\textasciigrave{}}\AttributeTok{\%in\%}\StringTok{\textasciigrave{}}\NormalTok{)}

\CommentTok{\#Preparando a base para as análises}
\CommentTok{\#Incluo na base apenas as categorias "Empreendedor ou Empregado (CNPJ)" e}
\CommentTok{\#"Empregado (CLT)" da variável "Qual sua situação atual de trabalho?".}
\CommentTok{\#Excluo a categoria de gênero "Outros".}

\NormalTok{df2}\OtherTok{=}\NormalTok{SoD2019}\SpecialCharTok{\%\textgreater{}\%}\NormalTok{dplyr}\SpecialCharTok{::}\FunctionTok{filter}\NormalTok{(}
  \StringTok{\textasciigrave{}}\AttributeTok{(\textquotesingle{}P10\textquotesingle{}, \textquotesingle{}job\_situation\textquotesingle{})}\StringTok{\textasciigrave{}}\SpecialCharTok{\%in\%}\FunctionTok{c}\NormalTok{(}
    \StringTok{\textquotesingle{}Empreendedor ou Empregado (CNPJ)\textquotesingle{}}\NormalTok{, }\StringTok{\textquotesingle{}Empregado (CTL)\textquotesingle{}}\NormalTok{) }\SpecialCharTok{\&}
    \StringTok{\textasciigrave{}}\AttributeTok{(\textquotesingle{}P2\textquotesingle{}, \textquotesingle{}gender\textquotesingle{})}\StringTok{\textasciigrave{}}\SpecialCharTok{!=}\StringTok{\textquotesingle{}Outro\textquotesingle{}}\NormalTok{ )}

\CommentTok{\#Pessoas por gênero {-} datahackers{-}survey{-}2019}

\NormalTok{df2}\SpecialCharTok{\%\textgreater{}\%}\NormalTok{dplyr}\SpecialCharTok{::}\FunctionTok{count}\NormalTok{(}\StringTok{\textasciigrave{}}\AttributeTok{(\textquotesingle{}P2\textquotesingle{}, \textquotesingle{}gender\textquotesingle{})}\StringTok{\textasciigrave{}}\NormalTok{)}\SpecialCharTok{\%\textgreater{}\%}
\NormalTok{  dplyr}\SpecialCharTok{::}\FunctionTok{mutate}\NormalTok{(}\AttributeTok{percentual =}\NormalTok{ n}\SpecialCharTok{/}\FunctionTok{sum}\NormalTok{(n)}\SpecialCharTok{*}\DecValTok{100}\NormalTok{)}\SpecialCharTok{\%\textgreater{}\%}\NormalTok{.[,}\SpecialCharTok{{-}}\DecValTok{2}\NormalTok{]}
\end{Highlighting}
\end{Shaded}

\begin{verbatim}
## # A tibble: 2 x 2
##   `('P2', 'gender')` percentual
##   <chr>                   <dbl>
## 1 Feminino                 17.3
## 2 Masculino                82.7
\end{verbatim}

Investigando as funções exercidas pelos profissionais da área de acordo
com seu gênero, é possível perceber uma maior concentração das mulheres
entrevistadas nos cargos de analise de dados, com 46\% delas atuando
como analistas, enquanto que apenas 33\% dos homens entrevistados
exercem tal função. Considerando as 4 principais atuações mostradas na
pesquisa (análise de dados, ciência de dados, engenharia de dados e
gestor), esta é a que, de uma maneira geral, pior paga, segundo o
relatório do State of Data Brazil 2021 - Figura 28. Além disso, é
importante notar a menor presença relativa das mulheres entre os
gestores (16,3\% das mulheres entrevistadas são gestoras, enquanto que
este percentual entre os homens é de 24,4\%). Isso pode indicar uma
dificuldade que o gênero feminino estaria enfrentando nas empresas:
assumir cargos de maior responsabilidade e que requerem maior confiança
por parte dos superiores. Analisando os níveis de cargo em cada atuação
separadamente, encontra-se certo equilibrio na função análise de dados.
Já entre cientistas de dados, as mulheres são, proporcionalmente, mais
presentes nos cargos júnior (34,4\% x 22,9\%), menos no nível pleno
(35,9\% x 47,5\%) e igualmente presente no sênior (29\% x 29\%). O caso
mais crítico, no entanto, é na atuação de engenharia de dados, onde
apenas 12,5\% das mulheres estão em cargos sênior contra 40,3\% dos
homens.

\begin{Shaded}
\begin{Highlighting}[]
\CommentTok{\#Atuação por gênero}
\NormalTok{df}\SpecialCharTok{\%\textgreater{}\%}
\NormalTok{  dplyr}\SpecialCharTok{::}\FunctionTok{count}\NormalTok{(}\StringTok{\textasciigrave{}}\AttributeTok{(\textquotesingle{}P1\_b \textquotesingle{}, \textquotesingle{}Genero\textquotesingle{})}\StringTok{\textasciigrave{}}\NormalTok{, }\StringTok{\textasciigrave{}}\AttributeTok{(\textquotesingle{}P4\_a \textquotesingle{}, \textquotesingle{}Atuacao\textquotesingle{})}\StringTok{\textasciigrave{}}\NormalTok{) }\SpecialCharTok{\%\textgreater{}\%}
\NormalTok{  dplyr}\SpecialCharTok{::}\FunctionTok{group\_by}\NormalTok{(}\StringTok{\textasciigrave{}}\AttributeTok{(\textquotesingle{}P1\_b \textquotesingle{}, \textquotesingle{}Genero\textquotesingle{})}\StringTok{\textasciigrave{}}\NormalTok{) }\SpecialCharTok{\%\textgreater{}\%}
\NormalTok{  dplyr}\SpecialCharTok{::}\FunctionTok{mutate}\NormalTok{(}\AttributeTok{Prop =}\NormalTok{ n}\SpecialCharTok{/}\FunctionTok{sum}\NormalTok{(n))}\SpecialCharTok{\%\textgreater{}\%}
\NormalTok{  ggplot2}\SpecialCharTok{::}\FunctionTok{ggplot}\NormalTok{(}
\NormalTok{    ggplot2}\SpecialCharTok{::}\FunctionTok{aes}\NormalTok{(}\AttributeTok{x =} \StringTok{\textasciigrave{}}\AttributeTok{(\textquotesingle{}P1\_b \textquotesingle{}, \textquotesingle{}Genero\textquotesingle{})}\StringTok{\textasciigrave{}}\NormalTok{, }\AttributeTok{y =}\NormalTok{ Prop,}
                 \AttributeTok{fill =} \StringTok{\textasciigrave{}}\AttributeTok{(\textquotesingle{}P4\_a \textquotesingle{}, \textquotesingle{}Atuacao\textquotesingle{})}\StringTok{\textasciigrave{}}\NormalTok{)) }\SpecialCharTok{+}
\NormalTok{  ggplot2}\SpecialCharTok{::}\FunctionTok{geom\_col}\NormalTok{( }\AttributeTok{color=}\StringTok{"white"}\NormalTok{,}
                     \AttributeTok{position =}\NormalTok{ ggplot2}\SpecialCharTok{::}\FunctionTok{position\_fill}\NormalTok{()) }\SpecialCharTok{+}
\NormalTok{  ggrepel}\SpecialCharTok{::}\FunctionTok{geom\_label\_repel}\NormalTok{(ggplot2}\SpecialCharTok{::}\FunctionTok{aes}\NormalTok{(}
    \AttributeTok{label =}\NormalTok{ scales}\SpecialCharTok{::}\FunctionTok{percent}\NormalTok{(Prop)),}
    \AttributeTok{fontface =} \StringTok{\textquotesingle{}bold\textquotesingle{}}\NormalTok{,}
    \AttributeTok{hjust=}\DecValTok{2}\NormalTok{,}
    \AttributeTok{position =}\NormalTok{ ggplot2}\SpecialCharTok{::}\FunctionTok{position\_stack}\NormalTok{(}\AttributeTok{vjust =}\NormalTok{ .}\DecValTok{5}\NormalTok{),}
    \AttributeTok{size=}\FloatTok{3.5}\NormalTok{) }\SpecialCharTok{+}
\NormalTok{  ggplot2}\SpecialCharTok{::}\FunctionTok{ggtitle}\NormalTok{(}\StringTok{\textquotesingle{}Figura 1: Atuação x Gênero\textquotesingle{}}\NormalTok{)}\SpecialCharTok{+}
\NormalTok{  ggplot2}\SpecialCharTok{::}\FunctionTok{theme\_void}\NormalTok{()}\SpecialCharTok{+}
\NormalTok{  ggplot2}\SpecialCharTok{::}\FunctionTok{scale\_fill\_brewer}\NormalTok{(}\AttributeTok{type =} \StringTok{"seq"}\NormalTok{, }\AttributeTok{palette =} \StringTok{"Spectral"}\NormalTok{)}\SpecialCharTok{+}
\NormalTok{  ggplot2}\SpecialCharTok{::}\FunctionTok{xlab}\NormalTok{(}\StringTok{\textquotesingle{}Gênero\textquotesingle{}}\NormalTok{)}\SpecialCharTok{+}
\NormalTok{  ggplot2}\SpecialCharTok{::}\FunctionTok{theme}\NormalTok{(}\AttributeTok{legend.position=}\StringTok{"bottom"}\NormalTok{,}
                 \AttributeTok{plot.title=}\NormalTok{ggplot2}\SpecialCharTok{::}\FunctionTok{element\_text}\NormalTok{(}\AttributeTok{face=}\StringTok{\textquotesingle{}bold.italic\textquotesingle{}}\NormalTok{,}
                                                  \AttributeTok{hjust =} \FloatTok{0.5}\NormalTok{, }\AttributeTok{size=}\DecValTok{20}\NormalTok{),}
                 \AttributeTok{axis.text.y=}\NormalTok{ggplot2}\SpecialCharTok{::}\FunctionTok{element\_blank}\NormalTok{(),}
                 \AttributeTok{axis.title.y=}\NormalTok{ggplot2}\SpecialCharTok{::}\FunctionTok{element\_blank}\NormalTok{(),}
                 \AttributeTok{axis.title.x=}\NormalTok{ggplot2}\SpecialCharTok{::}\FunctionTok{element\_blank}\NormalTok{(),}
                 \AttributeTok{axis.text.x =}\NormalTok{ggplot2}\SpecialCharTok{::}\FunctionTok{element\_text}\NormalTok{(}\AttributeTok{face=}\StringTok{\textquotesingle{}bold\textquotesingle{}}\NormalTok{, }\AttributeTok{size=}\DecValTok{12}\NormalTok{),}
                 \AttributeTok{legend.title=}\NormalTok{ggplot2}\SpecialCharTok{::}\FunctionTok{element\_blank}\NormalTok{())}
\end{Highlighting}
\end{Shaded}

\includegraphics{relatorio_files/figure-latex/unnamed-chunk-3-1.pdf}

\begin{Shaded}
\begin{Highlighting}[]
\CommentTok{\#Nível de cargo por gênero e atuação}
\NormalTok{df}\SpecialCharTok{\%\textgreater{}\%}
\NormalTok{  dplyr}\SpecialCharTok{::}\FunctionTok{count}\NormalTok{(}\StringTok{\textasciigrave{}}\AttributeTok{(\textquotesingle{}P4\_a \textquotesingle{}, \textquotesingle{}Atuacao\textquotesingle{})}\StringTok{\textasciigrave{}}\NormalTok{,}\StringTok{\textasciigrave{}}\AttributeTok{(\textquotesingle{}P1\_b \textquotesingle{}, \textquotesingle{}Genero\textquotesingle{})}\StringTok{\textasciigrave{}}\NormalTok{, }\StringTok{\textasciigrave{}}\AttributeTok{(\textquotesingle{}P2\_g \textquotesingle{}, \textquotesingle{}Nivel\textquotesingle{})}\StringTok{\textasciigrave{}}\NormalTok{) }\SpecialCharTok{\%\textgreater{}\%}
\NormalTok{  dplyr}\SpecialCharTok{::}\FunctionTok{group\_by}\NormalTok{(}\StringTok{\textasciigrave{}}\AttributeTok{(\textquotesingle{}P1\_b \textquotesingle{}, \textquotesingle{}Genero\textquotesingle{})}\StringTok{\textasciigrave{}}\NormalTok{,}\StringTok{\textasciigrave{}}\AttributeTok{(\textquotesingle{}P4\_a \textquotesingle{}, \textquotesingle{}Atuacao\textquotesingle{})}\StringTok{\textasciigrave{}}\NormalTok{) }\SpecialCharTok{\%\textgreater{}\%}
\NormalTok{  dplyr}\SpecialCharTok{::}\FunctionTok{mutate}\NormalTok{(}\AttributeTok{Prop =}\NormalTok{ n}\SpecialCharTok{/}\FunctionTok{sum}\NormalTok{(n))}\SpecialCharTok{\%\textgreater{}\%}
\NormalTok{  dplyr}\SpecialCharTok{::}\FunctionTok{filter}\NormalTok{(}\StringTok{\textasciigrave{}}\AttributeTok{(\textquotesingle{}P4\_a \textquotesingle{}, \textquotesingle{}Atuacao\textquotesingle{})}\StringTok{\textasciigrave{}}\SpecialCharTok{\%notin\%}\FunctionTok{c}\NormalTok{(}\StringTok{\textquotesingle{}Gestor\textquotesingle{}}\NormalTok{, }\StringTok{\textquotesingle{}Outra\textquotesingle{}}\NormalTok{))}\SpecialCharTok{\%\textgreater{}\%}
\NormalTok{  ggplot2}\SpecialCharTok{::}\FunctionTok{ggplot}\NormalTok{(}
\NormalTok{    ggplot2}\SpecialCharTok{::}\FunctionTok{aes}\NormalTok{(}\AttributeTok{x =} \StringTok{\textasciigrave{}}\AttributeTok{(\textquotesingle{}P1\_b \textquotesingle{}, \textquotesingle{}Genero\textquotesingle{})}\StringTok{\textasciigrave{}}\NormalTok{, }\AttributeTok{y =}\NormalTok{ Prop,}
                 \AttributeTok{fill =} \StringTok{\textasciigrave{}}\AttributeTok{(\textquotesingle{}P2\_g \textquotesingle{}, \textquotesingle{}Nivel\textquotesingle{})}\StringTok{\textasciigrave{}}\NormalTok{)) }\SpecialCharTok{+}
\NormalTok{  ggplot2}\SpecialCharTok{::}\FunctionTok{geom\_col}\NormalTok{( }\AttributeTok{color=}\StringTok{"white"}\NormalTok{,}
                     \AttributeTok{position =}\NormalTok{ ggplot2}\SpecialCharTok{::}\FunctionTok{position\_fill}\NormalTok{()) }\SpecialCharTok{+}
\NormalTok{  ggrepel}\SpecialCharTok{::}\FunctionTok{geom\_label\_repel}\NormalTok{(ggplot2}\SpecialCharTok{::}\FunctionTok{aes}\NormalTok{(}
    \AttributeTok{label =}\NormalTok{ scales}\SpecialCharTok{::}\FunctionTok{percent}\NormalTok{(Prop)),}
    \AttributeTok{fontface =} \StringTok{\textquotesingle{}bold\textquotesingle{}}\NormalTok{,}
    \AttributeTok{hjust=}\DecValTok{2}\NormalTok{,}
    \AttributeTok{position =}\NormalTok{ ggplot2}\SpecialCharTok{::}\FunctionTok{position\_stack}\NormalTok{(}\AttributeTok{vjust =}\NormalTok{ .}\DecValTok{5}\NormalTok{),}
    \AttributeTok{size=}\FloatTok{3.5}\NormalTok{) }\SpecialCharTok{+}
\NormalTok{  ggplot2}\SpecialCharTok{::}\FunctionTok{ggtitle}\NormalTok{(}\StringTok{\textquotesingle{}Figura 2: Nível de Cargo x Atuação x Gênero\textquotesingle{}}\NormalTok{)}\SpecialCharTok{+}
\NormalTok{  ggplot2}\SpecialCharTok{::}\FunctionTok{theme\_void}\NormalTok{()}\SpecialCharTok{+}
\NormalTok{  ggplot2}\SpecialCharTok{::}\FunctionTok{scale\_fill\_brewer}\NormalTok{(}\AttributeTok{type =} \StringTok{"seq"}\NormalTok{, }\AttributeTok{palette =} \StringTok{"Oranges"}\NormalTok{)}\SpecialCharTok{+}
\NormalTok{  ggplot2}\SpecialCharTok{::}\FunctionTok{xlab}\NormalTok{(}\StringTok{\textquotesingle{}Gênero\textquotesingle{}}\NormalTok{)}\SpecialCharTok{+}
\NormalTok{  ggplot2}\SpecialCharTok{::}\FunctionTok{theme}\NormalTok{(}\AttributeTok{legend.position=}\StringTok{"bottom"}\NormalTok{,}
                 \AttributeTok{plot.title=}\NormalTok{ggplot2}\SpecialCharTok{::}\FunctionTok{element\_text}\NormalTok{(}\AttributeTok{face=}\StringTok{\textquotesingle{}bold.italic\textquotesingle{}}\NormalTok{,}
                                                  \AttributeTok{hjust =} \FloatTok{0.5}\NormalTok{, }\AttributeTok{size=}\DecValTok{20}\NormalTok{),}
                 \AttributeTok{axis.text.y=}\NormalTok{ggplot2}\SpecialCharTok{::}\FunctionTok{element\_blank}\NormalTok{(),}
                 \AttributeTok{axis.title.y=}\NormalTok{ggplot2}\SpecialCharTok{::}\FunctionTok{element\_blank}\NormalTok{(),}
                 \AttributeTok{axis.title.x=}\NormalTok{ggplot2}\SpecialCharTok{::}\FunctionTok{element\_blank}\NormalTok{(),}
                 \AttributeTok{axis.text.x =}\NormalTok{ggplot2}\SpecialCharTok{::}\FunctionTok{element\_text}\NormalTok{(}\AttributeTok{face=}\StringTok{\textquotesingle{}bold\textquotesingle{}}\NormalTok{, }\AttributeTok{size=}\DecValTok{12}\NormalTok{),}
                 \AttributeTok{legend.title=}\NormalTok{ggplot2}\SpecialCharTok{::}\FunctionTok{element\_blank}\NormalTok{())}\SpecialCharTok{+}
\NormalTok{  ggplot2}\SpecialCharTok{::}\FunctionTok{facet\_grid}\NormalTok{(}\SpecialCharTok{\textasciitilde{}}\StringTok{\textasciigrave{}}\AttributeTok{(\textquotesingle{}P4\_a \textquotesingle{}, \textquotesingle{}Atuacao\textquotesingle{})}\StringTok{\textasciigrave{}}\NormalTok{)}
\end{Highlighting}
\end{Shaded}

\includegraphics{relatorio_files/figure-latex/unnamed-chunk-3-2.pdf} O
curioso é o fato das mulheres serem mais escolarizadas que os homens e
quem diz isso é o IBGE. Os dados constam da segunda edição do estudo
``Estatísticas de gênero: indicadores sociais das mulheres no Brasil''.
Ele traz informações variadas sobre as condições de vida das brasileiras
em 2019
\url{https://epocanegocios.globo.com/Economia/noticia/2021/03/epoca-negocios-mulheres-sao-mais-instruidas-mas-ocupam-apenas-374-dos-cargos-gerenciais.html}
. O estudo ainda trás outros dados, como a maior dificuldade das
mulheres em acessar cargos de chefia e gerência (corroborando com a
Figura 1), além do rendimento médio inferior a 80\% do rendimento dos
homens (de acordo com o relatório do State of Data Brazil 2021 - Figura
26, as mulheres da área de dados estão ganhando menos nos cargos sênior
e de gestão). Ao que tudo indica, também é possível verificar isso na
área de dados, onde, considerando as funções de análise de dados e
engenharia de dados, a proporção das mulheres entrevistadas que não
possuem o nível de pós-graduação é menor que a dos homens na mesma
situação, já entre os cientistas de dados, há mais equilíbrio. Isto
também acontece entre os níveis de cargo, sendo que, considerando os
profissionais sênior, entre as mulheres apenas 22,1\% não possuem
pós-graduação, enquanto que este percentual é de mais de 35.7\% entre os
homens. Já em relação aos gestores, no entanto, há mais equilíbrio.
Entre as gestoras entrevistadas apenas 31,15\% não possuem
pós-graduação, enquanto que este percentual é de cerca de 35.8\% entre
os homens. Além disso, em termos de áreas de formação, as mulheres
entrevistadas estão menos concentradas no campo da TI e afins do que os
homens (34,5\% x 45,2\%), resultando em uma maior diversidade de
formaturas, estando proporcionalmente mais presente em outras áreas do
conhecimento, como estatística, área de saúde, marketing, etc.

\begin{Shaded}
\begin{Highlighting}[]
\CommentTok{\#Nível de ensino por gênero e atuação}
\NormalTok{df}\SpecialCharTok{\%\textgreater{}\%}
\NormalTok{  dplyr}\SpecialCharTok{::}\FunctionTok{count}\NormalTok{(}\StringTok{\textasciigrave{}}\AttributeTok{(\textquotesingle{}P4\_a \textquotesingle{}, \textquotesingle{}Atuacao\textquotesingle{})}\StringTok{\textasciigrave{}}\NormalTok{,}\StringTok{\textasciigrave{}}\AttributeTok{(\textquotesingle{}P1\_b \textquotesingle{}, \textquotesingle{}Genero\textquotesingle{})}\StringTok{\textasciigrave{}}\NormalTok{, }\StringTok{\textasciigrave{}}\AttributeTok{(\textquotesingle{}P1\_h \textquotesingle{}, \textquotesingle{}Nivel de Ensino\textquotesingle{})}\StringTok{\textasciigrave{}}\NormalTok{)}\SpecialCharTok{\%\textgreater{}\%}
\NormalTok{  dplyr}\SpecialCharTok{::}\FunctionTok{group\_by}\NormalTok{(}\StringTok{\textasciigrave{}}\AttributeTok{(\textquotesingle{}P4\_a \textquotesingle{}, \textquotesingle{}Atuacao\textquotesingle{})}\StringTok{\textasciigrave{}}\NormalTok{,}\StringTok{\textasciigrave{}}\AttributeTok{(\textquotesingle{}P1\_b \textquotesingle{}, \textquotesingle{}Genero\textquotesingle{})}\StringTok{\textasciigrave{}}\NormalTok{) }\SpecialCharTok{\%\textgreater{}\%}
\NormalTok{  dplyr}\SpecialCharTok{::}\FunctionTok{mutate}\NormalTok{(}\AttributeTok{Prop =}\NormalTok{ n}\SpecialCharTok{/}\FunctionTok{sum}\NormalTok{(n))}\SpecialCharTok{\%\textgreater{}\%}
\NormalTok{  dplyr}\SpecialCharTok{::}\FunctionTok{filter}\NormalTok{(}\StringTok{\textasciigrave{}}\AttributeTok{(\textquotesingle{}P4\_a \textquotesingle{}, \textquotesingle{}Atuacao\textquotesingle{})}\StringTok{\textasciigrave{}}\SpecialCharTok{\%notin\%}\FunctionTok{c}\NormalTok{(}\StringTok{\textquotesingle{}Outra\textquotesingle{}}\NormalTok{, }\StringTok{\textquotesingle{}Gestor\textquotesingle{}}\NormalTok{))}\SpecialCharTok{\%\textgreater{}\%}
\NormalTok{  ggplot2}\SpecialCharTok{::}\FunctionTok{ggplot}\NormalTok{(}
\NormalTok{    ggplot2}\SpecialCharTok{::}\FunctionTok{aes}\NormalTok{(}\AttributeTok{x =} \StringTok{\textasciigrave{}}\AttributeTok{(\textquotesingle{}P1\_b \textquotesingle{}, \textquotesingle{}Genero\textquotesingle{})}\StringTok{\textasciigrave{}}\NormalTok{, }\AttributeTok{y =}\NormalTok{ Prop,}
                 \AttributeTok{fill =} \StringTok{\textasciigrave{}}\AttributeTok{(\textquotesingle{}P1\_h \textquotesingle{}, \textquotesingle{}Nivel de Ensino\textquotesingle{})}\StringTok{\textasciigrave{}}\NormalTok{)) }\SpecialCharTok{+}
\NormalTok{  ggplot2}\SpecialCharTok{::}\FunctionTok{geom\_col}\NormalTok{( }\AttributeTok{color=}\StringTok{"white"}\NormalTok{,}
                     \AttributeTok{position =}\NormalTok{ ggplot2}\SpecialCharTok{::}\FunctionTok{position\_fill}\NormalTok{()) }\SpecialCharTok{+}
\NormalTok{  ggrepel}\SpecialCharTok{::}\FunctionTok{geom\_label\_repel}\NormalTok{(ggplot2}\SpecialCharTok{::}\FunctionTok{aes}\NormalTok{(}
    \AttributeTok{label =}\NormalTok{ scales}\SpecialCharTok{::}\FunctionTok{percent}\NormalTok{(Prop)),}
    \AttributeTok{fontface =} \StringTok{\textquotesingle{}bold\textquotesingle{}}\NormalTok{,}
    \AttributeTok{hjust=}\DecValTok{2}\NormalTok{,}
    \AttributeTok{position =}\NormalTok{ ggplot2}\SpecialCharTok{::}\FunctionTok{position\_stack}\NormalTok{(}\AttributeTok{vjust =}\NormalTok{ .}\DecValTok{5}\NormalTok{),}
    \AttributeTok{size=}\FloatTok{3.5}\NormalTok{) }\SpecialCharTok{+}
\NormalTok{  ggplot2}\SpecialCharTok{::}\FunctionTok{ggtitle}\NormalTok{(}\StringTok{\textquotesingle{}Figura 3: Nível de Ensino x Atuação x Gênero\textquotesingle{}}\NormalTok{)}\SpecialCharTok{+}
\NormalTok{  ggplot2}\SpecialCharTok{::}\FunctionTok{theme\_void}\NormalTok{()}\SpecialCharTok{+}
\NormalTok{  ggplot2}\SpecialCharTok{::}\FunctionTok{scale\_fill\_brewer}\NormalTok{(}\AttributeTok{type =} \StringTok{"seq"}\NormalTok{, }\AttributeTok{palette =} \StringTok{"BuPu"}\NormalTok{)}\SpecialCharTok{+}
\NormalTok{  ggplot2}\SpecialCharTok{::}\FunctionTok{xlab}\NormalTok{(}\StringTok{\textquotesingle{}Gênero\textquotesingle{}}\NormalTok{)}\SpecialCharTok{+}
\NormalTok{  ggplot2}\SpecialCharTok{::}\FunctionTok{theme}\NormalTok{(}\AttributeTok{legend.position=}\StringTok{"bottom"}\NormalTok{,}
                 \AttributeTok{plot.title=}\NormalTok{ggplot2}\SpecialCharTok{::}\FunctionTok{element\_text}\NormalTok{(}\AttributeTok{face=}\StringTok{\textquotesingle{}bold.italic\textquotesingle{}}\NormalTok{,}
                                                  \AttributeTok{hjust =} \FloatTok{0.5}\NormalTok{, }\AttributeTok{size=}\DecValTok{20}\NormalTok{),}
                 \AttributeTok{axis.text.y=}\NormalTok{ggplot2}\SpecialCharTok{::}\FunctionTok{element\_blank}\NormalTok{(),}
                 \AttributeTok{axis.title.y=}\NormalTok{ggplot2}\SpecialCharTok{::}\FunctionTok{element\_blank}\NormalTok{(),}
                 \AttributeTok{axis.title.x=}\NormalTok{ggplot2}\SpecialCharTok{::}\FunctionTok{element\_blank}\NormalTok{(),}
                 \AttributeTok{axis.text.x =}\NormalTok{ggplot2}\SpecialCharTok{::}\FunctionTok{element\_text}\NormalTok{(}\AttributeTok{face=}\StringTok{\textquotesingle{}bold\textquotesingle{}}\NormalTok{, }\AttributeTok{size=}\DecValTok{12}\NormalTok{),}
                 \AttributeTok{legend.title=}\NormalTok{ggplot2}\SpecialCharTok{::}\FunctionTok{element\_blank}\NormalTok{())}\SpecialCharTok{+}
\NormalTok{  ggplot2}\SpecialCharTok{::}\FunctionTok{facet\_grid}\NormalTok{(}\SpecialCharTok{\textasciitilde{}}\StringTok{\textasciigrave{}}\AttributeTok{(\textquotesingle{}P4\_a \textquotesingle{}, \textquotesingle{}Atuacao\textquotesingle{})}\StringTok{\textasciigrave{}}\NormalTok{)}
\end{Highlighting}
\end{Shaded}

\includegraphics{relatorio_files/figure-latex/unnamed-chunk-4-1.pdf}

\begin{Shaded}
\begin{Highlighting}[]
\CommentTok{\#Nível de ensino por gênero e Nível de cargo}
\NormalTok{df}\SpecialCharTok{\%\textgreater{}\%}
\NormalTok{  dplyr}\SpecialCharTok{::}\FunctionTok{count}\NormalTok{(}\StringTok{\textasciigrave{}}\AttributeTok{(\textquotesingle{}P2\_g \textquotesingle{}, \textquotesingle{}Nivel\textquotesingle{})}\StringTok{\textasciigrave{}}\NormalTok{,}\StringTok{\textasciigrave{}}\AttributeTok{(\textquotesingle{}P1\_b \textquotesingle{}, \textquotesingle{}Genero\textquotesingle{})}\StringTok{\textasciigrave{}}\NormalTok{, }\StringTok{\textasciigrave{}}\AttributeTok{(\textquotesingle{}P1\_h \textquotesingle{}, \textquotesingle{}Nivel de Ensino\textquotesingle{})}\StringTok{\textasciigrave{}}\NormalTok{)}\SpecialCharTok{\%\textgreater{}\%}
\NormalTok{  dplyr}\SpecialCharTok{::}\FunctionTok{group\_by}\NormalTok{(}\StringTok{\textasciigrave{}}\AttributeTok{(\textquotesingle{}P2\_g \textquotesingle{}, \textquotesingle{}Nivel\textquotesingle{})}\StringTok{\textasciigrave{}}\NormalTok{,}\StringTok{\textasciigrave{}}\AttributeTok{(\textquotesingle{}P1\_b \textquotesingle{}, \textquotesingle{}Genero\textquotesingle{})}\StringTok{\textasciigrave{}}\NormalTok{) }\SpecialCharTok{\%\textgreater{}\%}
\NormalTok{  dplyr}\SpecialCharTok{::}\FunctionTok{mutate}\NormalTok{(}\AttributeTok{Prop =}\NormalTok{ n}\SpecialCharTok{/}\FunctionTok{sum}\NormalTok{(n))}\SpecialCharTok{\%\textgreater{}\%}
\NormalTok{  dplyr}\SpecialCharTok{::}\FunctionTok{filter}\NormalTok{(}\SpecialCharTok{!}\FunctionTok{is.na}\NormalTok{(}\StringTok{\textasciigrave{}}\AttributeTok{(\textquotesingle{}P2\_g \textquotesingle{}, \textquotesingle{}Nivel\textquotesingle{})}\StringTok{\textasciigrave{}}\NormalTok{))}\SpecialCharTok{\%\textgreater{}\%}
\NormalTok{  ggplot2}\SpecialCharTok{::}\FunctionTok{ggplot}\NormalTok{(}
\NormalTok{    ggplot2}\SpecialCharTok{::}\FunctionTok{aes}\NormalTok{(}\AttributeTok{x =} \StringTok{\textasciigrave{}}\AttributeTok{(\textquotesingle{}P1\_b \textquotesingle{}, \textquotesingle{}Genero\textquotesingle{})}\StringTok{\textasciigrave{}}\NormalTok{, }\AttributeTok{y =}\NormalTok{ Prop,}
                 \AttributeTok{fill =} \StringTok{\textasciigrave{}}\AttributeTok{(\textquotesingle{}P1\_h \textquotesingle{}, \textquotesingle{}Nivel de Ensino\textquotesingle{})}\StringTok{\textasciigrave{}}\NormalTok{)) }\SpecialCharTok{+}
\NormalTok{  ggplot2}\SpecialCharTok{::}\FunctionTok{geom\_col}\NormalTok{( }\AttributeTok{color=}\StringTok{"white"}\NormalTok{,}
                     \AttributeTok{position =}\NormalTok{ ggplot2}\SpecialCharTok{::}\FunctionTok{position\_fill}\NormalTok{()) }\SpecialCharTok{+}
\NormalTok{  ggrepel}\SpecialCharTok{::}\FunctionTok{geom\_label\_repel}\NormalTok{(ggplot2}\SpecialCharTok{::}\FunctionTok{aes}\NormalTok{(}
    \AttributeTok{label =}\NormalTok{ scales}\SpecialCharTok{::}\FunctionTok{percent}\NormalTok{(Prop)),}
    \AttributeTok{fontface =} \StringTok{\textquotesingle{}bold\textquotesingle{}}\NormalTok{,}
    \AttributeTok{hjust=}\DecValTok{2}\NormalTok{,}
    \AttributeTok{position =}\NormalTok{ ggplot2}\SpecialCharTok{::}\FunctionTok{position\_stack}\NormalTok{(}\AttributeTok{vjust =}\NormalTok{ .}\DecValTok{5}\NormalTok{),}
    \AttributeTok{size=}\FloatTok{3.5}\NormalTok{) }\SpecialCharTok{+}
\NormalTok{  ggplot2}\SpecialCharTok{::}\FunctionTok{ggtitle}\NormalTok{(}\StringTok{\textquotesingle{}Figura 4: Nível de Ensino x Nível de Cargo x Gênero\textquotesingle{}}\NormalTok{)}\SpecialCharTok{+}
\NormalTok{  ggplot2}\SpecialCharTok{::}\FunctionTok{theme\_void}\NormalTok{()}\SpecialCharTok{+}
\NormalTok{  ggplot2}\SpecialCharTok{::}\FunctionTok{scale\_fill\_brewer}\NormalTok{(}\AttributeTok{type =} \StringTok{"seq"}\NormalTok{, }\AttributeTok{palette =} \StringTok{"BuPu"}\NormalTok{)}\SpecialCharTok{+}
\NormalTok{  ggplot2}\SpecialCharTok{::}\FunctionTok{xlab}\NormalTok{(}\StringTok{\textquotesingle{}Gênero\textquotesingle{}}\NormalTok{)}\SpecialCharTok{+}
\NormalTok{  ggplot2}\SpecialCharTok{::}\FunctionTok{theme}\NormalTok{(}\AttributeTok{legend.position=}\StringTok{"bottom"}\NormalTok{,}
                 \AttributeTok{plot.title=}\NormalTok{ggplot2}\SpecialCharTok{::}\FunctionTok{element\_text}\NormalTok{(}\AttributeTok{face=}\StringTok{\textquotesingle{}bold.italic\textquotesingle{}}\NormalTok{,}
                                                  \AttributeTok{hjust =} \FloatTok{0.5}\NormalTok{, }\AttributeTok{size=}\DecValTok{20}\NormalTok{),}
                 \AttributeTok{axis.text.y=}\NormalTok{ggplot2}\SpecialCharTok{::}\FunctionTok{element\_blank}\NormalTok{(),}
                 \AttributeTok{axis.title.y=}\NormalTok{ggplot2}\SpecialCharTok{::}\FunctionTok{element\_blank}\NormalTok{(),}
                 \AttributeTok{axis.title.x=}\NormalTok{ggplot2}\SpecialCharTok{::}\FunctionTok{element\_blank}\NormalTok{(),}
                 \AttributeTok{axis.text.x =}\NormalTok{ggplot2}\SpecialCharTok{::}\FunctionTok{element\_text}\NormalTok{(}\AttributeTok{face=}\StringTok{\textquotesingle{}bold\textquotesingle{}}\NormalTok{, }\AttributeTok{size=}\DecValTok{12}\NormalTok{),}
                 \AttributeTok{legend.title=}\NormalTok{ggplot2}\SpecialCharTok{::}\FunctionTok{element\_blank}\NormalTok{())}\SpecialCharTok{+}
\NormalTok{  ggplot2}\SpecialCharTok{::}\FunctionTok{facet\_grid}\NormalTok{(}\SpecialCharTok{\textasciitilde{}}\StringTok{\textasciigrave{}}\AttributeTok{(\textquotesingle{}P2\_g \textquotesingle{}, \textquotesingle{}Nivel\textquotesingle{})}\StringTok{\textasciigrave{}}\NormalTok{)}
\end{Highlighting}
\end{Shaded}

\includegraphics{relatorio_files/figure-latex/unnamed-chunk-4-2.pdf}

\begin{Shaded}
\begin{Highlighting}[]
\CommentTok{\#Formação por gênero}
\NormalTok{df}\SpecialCharTok{\%\textgreater{}\%}
\NormalTok{  dplyr}\SpecialCharTok{::}\FunctionTok{count}\NormalTok{(}\StringTok{\textasciigrave{}}\AttributeTok{(\textquotesingle{}P1\_b \textquotesingle{}, \textquotesingle{}Genero\textquotesingle{})}\StringTok{\textasciigrave{}}\NormalTok{, }\StringTok{\textasciigrave{}}\AttributeTok{(\textquotesingle{}P1\_i \textquotesingle{}, \textquotesingle{}Área de Formação\textquotesingle{})}\StringTok{\textasciigrave{}}\NormalTok{) }\SpecialCharTok{\%\textgreater{}\%}
\NormalTok{  dplyr}\SpecialCharTok{::}\FunctionTok{group\_by}\NormalTok{(}\StringTok{\textasciigrave{}}\AttributeTok{(\textquotesingle{}P1\_b \textquotesingle{}, \textquotesingle{}Genero\textquotesingle{})}\StringTok{\textasciigrave{}}\NormalTok{)}\SpecialCharTok{\%\textgreater{}\%}\NormalTok{dplyr}\SpecialCharTok{::}\FunctionTok{filter}\NormalTok{(}
    \SpecialCharTok{!}\FunctionTok{is.na}\NormalTok{(}\StringTok{\textasciigrave{}}\AttributeTok{(\textquotesingle{}P1\_i \textquotesingle{}, \textquotesingle{}Área de Formação\textquotesingle{})}\StringTok{\textasciigrave{}}\NormalTok{)) }\SpecialCharTok{\%\textgreater{}\%}
\NormalTok{  dplyr}\SpecialCharTok{::}\FunctionTok{mutate}\NormalTok{(}\AttributeTok{Prop =}\NormalTok{ n}\SpecialCharTok{/}\FunctionTok{sum}\NormalTok{(n))}\SpecialCharTok{\%\textgreater{}\%}
\NormalTok{  ggplot2}\SpecialCharTok{::}\FunctionTok{ggplot}\NormalTok{(}
\NormalTok{    ggplot2}\SpecialCharTok{::}\FunctionTok{aes}\NormalTok{(}\AttributeTok{x =} \StringTok{\textasciigrave{}}\AttributeTok{(\textquotesingle{}P1\_b \textquotesingle{}, \textquotesingle{}Genero\textquotesingle{})}\StringTok{\textasciigrave{}}\NormalTok{, }\AttributeTok{y =}\NormalTok{ Prop,}
                 \AttributeTok{fill =} \StringTok{\textasciigrave{}}\AttributeTok{(\textquotesingle{}P1\_i \textquotesingle{}, \textquotesingle{}Área de Formação\textquotesingle{})}\StringTok{\textasciigrave{}}\NormalTok{)) }\SpecialCharTok{+}
\NormalTok{  ggplot2}\SpecialCharTok{::}\FunctionTok{geom\_col}\NormalTok{( }\AttributeTok{color=}\StringTok{"white"}\NormalTok{,}
                     \AttributeTok{position =}\NormalTok{ ggplot2}\SpecialCharTok{::}\FunctionTok{position\_fill}\NormalTok{()) }\SpecialCharTok{+}
\NormalTok{  ggrepel}\SpecialCharTok{::}\FunctionTok{geom\_label\_repel}\NormalTok{(ggplot2}\SpecialCharTok{::}\FunctionTok{aes}\NormalTok{(}
    \AttributeTok{label =}\NormalTok{ scales}\SpecialCharTok{::}\FunctionTok{percent}\NormalTok{(Prop)),}
    \AttributeTok{fontface =} \StringTok{\textquotesingle{}bold\textquotesingle{}}\NormalTok{,}
    \AttributeTok{hjust=}\DecValTok{2}\NormalTok{,}
    \AttributeTok{position =}\NormalTok{ ggplot2}\SpecialCharTok{::}\FunctionTok{position\_stack}\NormalTok{(}\AttributeTok{vjust =}\NormalTok{ .}\DecValTok{5}\NormalTok{),}
    \AttributeTok{size=}\FloatTok{3.5}\NormalTok{) }\SpecialCharTok{+}
\NormalTok{  ggplot2}\SpecialCharTok{::}\FunctionTok{labs}\NormalTok{(}\AttributeTok{title =} \StringTok{\textquotesingle{}Figura 5: Formação x Gênero\textquotesingle{}}\NormalTok{)}\SpecialCharTok{+}
\NormalTok{  ggplot2}\SpecialCharTok{::}\FunctionTok{theme\_void}\NormalTok{()}\SpecialCharTok{+}
\NormalTok{  ggplot2}\SpecialCharTok{::}\FunctionTok{scale\_fill\_brewer}\NormalTok{(}\AttributeTok{type =} \StringTok{"seq"}\NormalTok{, }\AttributeTok{palette =} \StringTok{"Pastel1"}\NormalTok{)}\SpecialCharTok{+}
\NormalTok{  ggplot2}\SpecialCharTok{::}\FunctionTok{xlab}\NormalTok{(}\StringTok{\textquotesingle{}Gênero\textquotesingle{}}\NormalTok{)}\SpecialCharTok{+}
\NormalTok{  ggplot2}\SpecialCharTok{::}\FunctionTok{theme}\NormalTok{(}\AttributeTok{legend.position=}\StringTok{"bottom"}\NormalTok{,}
                 \AttributeTok{plot.title=}\NormalTok{ggplot2}\SpecialCharTok{::}\FunctionTok{element\_text}\NormalTok{(}\AttributeTok{face=}\StringTok{\textquotesingle{}bold.italic\textquotesingle{}}\NormalTok{,}
                                                  \AttributeTok{hjust =} \FloatTok{0.5}\NormalTok{, }\AttributeTok{size=}\DecValTok{20}\NormalTok{),}
                 \AttributeTok{axis.text.y=}\NormalTok{ggplot2}\SpecialCharTok{::}\FunctionTok{element\_blank}\NormalTok{(),}
                 \AttributeTok{axis.title.y=}\NormalTok{ggplot2}\SpecialCharTok{::}\FunctionTok{element\_blank}\NormalTok{(),}
                 \AttributeTok{axis.title.x=}\NormalTok{ggplot2}\SpecialCharTok{::}\FunctionTok{element\_blank}\NormalTok{(),}
                 \AttributeTok{axis.text.x =}\NormalTok{ggplot2}\SpecialCharTok{::}\FunctionTok{element\_text}\NormalTok{(}\AttributeTok{face=}\StringTok{\textquotesingle{}bold\textquotesingle{}}\NormalTok{, }\AttributeTok{size=}\DecValTok{12}\NormalTok{),}
                 \AttributeTok{legend.title=}\NormalTok{ggplot2}\SpecialCharTok{::}\FunctionTok{element\_blank}\NormalTok{())}
\end{Highlighting}
\end{Shaded}

\includegraphics{relatorio_files/figure-latex/unnamed-chunk-4-3.pdf}
Entretanto, parece que o nível de escolaridade das mulheres ``vale
menos'' que o dos homens, pois considerando os 4 principais níveis
escolares (bacharel, pós, mestrado e doutorado) as mulheres que os
possuem estão menos presentes, proporcionalmente, nos cargos sênior e de
gestão, do que eles.

\begin{Shaded}
\begin{Highlighting}[]
\CommentTok{\#Nível de cargo por gênero e nível de ensino}
\NormalTok{df}\SpecialCharTok{\%\textgreater{}\%}
\NormalTok{       dplyr}\SpecialCharTok{::}\FunctionTok{count}\NormalTok{(}\StringTok{\textasciigrave{}}\AttributeTok{(\textquotesingle{}P2\_g \textquotesingle{}, \textquotesingle{}Nivel\textquotesingle{})}\StringTok{\textasciigrave{}}\NormalTok{,}\StringTok{\textasciigrave{}}\AttributeTok{(\textquotesingle{}P1\_b \textquotesingle{}, \textquotesingle{}Genero\textquotesingle{})}\StringTok{\textasciigrave{}}\NormalTok{,}
                    \StringTok{\textasciigrave{}}\AttributeTok{(\textquotesingle{}P1\_h \textquotesingle{}, \textquotesingle{}Nivel de Ensino\textquotesingle{})}\StringTok{\textasciigrave{}}\NormalTok{)}\SpecialCharTok{\%\textgreater{}\%}
\NormalTok{  dplyr}\SpecialCharTok{::}\FunctionTok{group\_by}\NormalTok{(}\StringTok{\textasciigrave{}}\AttributeTok{(\textquotesingle{}P1\_h \textquotesingle{}, \textquotesingle{}Nivel de Ensino\textquotesingle{})}\StringTok{\textasciigrave{}}\NormalTok{,}\StringTok{\textasciigrave{}}\AttributeTok{(\textquotesingle{}P1\_b \textquotesingle{}, \textquotesingle{}Genero\textquotesingle{})}\StringTok{\textasciigrave{}}\NormalTok{) }\SpecialCharTok{\%\textgreater{}\%}
\NormalTok{  dplyr}\SpecialCharTok{::}\FunctionTok{mutate}\NormalTok{(}\AttributeTok{Prop =}\NormalTok{ n}\SpecialCharTok{/}\FunctionTok{sum}\NormalTok{(n))}\SpecialCharTok{\%\textgreater{}\%}
\NormalTok{  dplyr}\SpecialCharTok{::}\FunctionTok{filter}\NormalTok{(}\SpecialCharTok{!}\FunctionTok{is.na}\NormalTok{(}\StringTok{\textasciigrave{}}\AttributeTok{(\textquotesingle{}P2\_g \textquotesingle{}, \textquotesingle{}Nivel\textquotesingle{})}\StringTok{\textasciigrave{}}\NormalTok{) }\SpecialCharTok{\&}
                  \StringTok{\textasciigrave{}}\AttributeTok{(\textquotesingle{}P1\_h \textquotesingle{}, \textquotesingle{}Nivel de Ensino\textquotesingle{})}\StringTok{\textasciigrave{}}\SpecialCharTok{\%notin\%}\FunctionTok{c}\NormalTok{(}\StringTok{\textquotesingle{}Prefiro não informar\textquotesingle{}}\NormalTok{, }\StringTok{\textquotesingle{}Não tenho graduação formal\textquotesingle{}}\NormalTok{, }\StringTok{\textquotesingle{}Estudante de Graduação\textquotesingle{}}\NormalTok{))}\SpecialCharTok{\%\textgreater{}\%}
\NormalTok{  ggplot2}\SpecialCharTok{::}\FunctionTok{ggplot}\NormalTok{(ggplot2}\SpecialCharTok{::}\FunctionTok{aes}\NormalTok{(}\AttributeTok{x =} \StringTok{\textasciigrave{}}\AttributeTok{(\textquotesingle{}P1\_b \textquotesingle{}, \textquotesingle{}Genero\textquotesingle{})}\StringTok{\textasciigrave{}}\NormalTok{, }\AttributeTok{y =}\NormalTok{ Prop,}
                      \AttributeTok{fill =} \StringTok{\textasciigrave{}}\AttributeTok{(\textquotesingle{}P2\_g \textquotesingle{}, \textquotesingle{}Nivel\textquotesingle{})}\StringTok{\textasciigrave{}}\NormalTok{)) }\SpecialCharTok{+}\NormalTok{ggplot2}\SpecialCharTok{::}\FunctionTok{geom\_col}\NormalTok{(}
                        \AttributeTok{color=}\StringTok{"white"}\NormalTok{, }\AttributeTok{position =}\NormalTok{ ggplot2}\SpecialCharTok{::}\FunctionTok{position\_fill}\NormalTok{()) }\SpecialCharTok{+}
\NormalTok{ggrepel}\SpecialCharTok{::}\FunctionTok{geom\_label\_repel}\NormalTok{(ggplot2}\SpecialCharTok{::}\FunctionTok{aes}\NormalTok{(}\AttributeTok{label =}\NormalTok{ scales}\SpecialCharTok{::}\FunctionTok{percent}\NormalTok{(Prop)),}
                          \AttributeTok{fontface =} \StringTok{\textquotesingle{}bold\textquotesingle{}}\NormalTok{, }\AttributeTok{hjust=}\DecValTok{2}\NormalTok{,}
                          \AttributeTok{position =}\NormalTok{ ggplot2}\SpecialCharTok{::}\FunctionTok{position\_stack}\NormalTok{(}\AttributeTok{vjust =}\NormalTok{ .}\DecValTok{5}\NormalTok{),}
                          \AttributeTok{size=}\FloatTok{3.5}\NormalTok{) }\SpecialCharTok{+}
\NormalTok{  ggplot2}\SpecialCharTok{::}\FunctionTok{ggtitle}\NormalTok{(}\StringTok{\textquotesingle{}Figura 6: Nível de Cargo x Nível de Ensino x Gênero\textquotesingle{}}\NormalTok{)}\SpecialCharTok{+}
\NormalTok{  ggplot2}\SpecialCharTok{::}\FunctionTok{theme\_void}\NormalTok{()}\SpecialCharTok{+}
\NormalTok{  ggplot2}\SpecialCharTok{::}\FunctionTok{scale\_fill\_brewer}\NormalTok{(}\AttributeTok{type =} \StringTok{"seq"}\NormalTok{, }\AttributeTok{palette =} \StringTok{"Oranges"}\NormalTok{)}\SpecialCharTok{+}
\NormalTok{  ggplot2}\SpecialCharTok{::}\FunctionTok{xlab}\NormalTok{(}\StringTok{\textquotesingle{}Gênero\textquotesingle{}}\NormalTok{)}\SpecialCharTok{+}
\NormalTok{  ggplot2}\SpecialCharTok{::}\FunctionTok{theme}\NormalTok{(}\AttributeTok{legend.position=}\StringTok{"bottom"}\NormalTok{,}
                 \AttributeTok{plot.title=}\NormalTok{ggplot2}\SpecialCharTok{::}\FunctionTok{element\_text}\NormalTok{(}\AttributeTok{face=}\StringTok{\textquotesingle{}bold.italic\textquotesingle{}}\NormalTok{),}
                 \AttributeTok{axis.text.y=}\NormalTok{ggplot2}\SpecialCharTok{::}\FunctionTok{element\_blank}\NormalTok{(),}
                 \AttributeTok{axis.title.y=}\NormalTok{ggplot2}\SpecialCharTok{::}\FunctionTok{element\_blank}\NormalTok{(),}
                 \AttributeTok{axis.title.x=}\NormalTok{ggplot2}\SpecialCharTok{::}\FunctionTok{element\_blank}\NormalTok{(),}
                 \AttributeTok{axis.text.x =}\NormalTok{ggplot2}\SpecialCharTok{::}\FunctionTok{element\_text}\NormalTok{(}\AttributeTok{face=}\StringTok{\textquotesingle{}bold\textquotesingle{}}\NormalTok{, }\AttributeTok{size=}\DecValTok{12}\NormalTok{),}
                 \AttributeTok{legend.title=}\NormalTok{ggplot2}\SpecialCharTok{::}\FunctionTok{element\_blank}\NormalTok{())}\SpecialCharTok{+}
\NormalTok{                   ggplot2}\SpecialCharTok{::}\FunctionTok{facet\_grid}\NormalTok{(}\SpecialCharTok{\textasciitilde{}}\StringTok{\textasciigrave{}}\AttributeTok{(\textquotesingle{}P1\_h \textquotesingle{}, \textquotesingle{}Nivel de Ensino\textquotesingle{})}\StringTok{\textasciigrave{}}\NormalTok{)}
\end{Highlighting}
\end{Shaded}

\includegraphics{relatorio_files/figure-latex/unnamed-chunk-5-1.pdf}

Em contrapartida, as mulheres entrevistadas mostraram ser, de uma
maneira geral, menos experientes, seja em relação à própria área de
dados ou, principalmente, em relação a vivências anteriores em alguma
outra área de TI. Em relação a área de dados, cerca de 35,7\% das
mulheres entrevistadas têm 4 ou mais anos de experiência, enquanto que
41,4\% dos homens estão na mesma situação. Em relação a experiências
prévias em outra áreas de TI, a diferença é maior. Menos de 14\% das
mulheres entrevistadas possuem 4 ou mais anos de experiência, enquanto
que, entre os homens, este percentual é superior a 25\%. Mais de 49\%
das mulheres disseram não ter experiência prévia em TI, enquanto que
quase 37\% dos homens entrevistados disseram o mesmo. Segundo o
relatório do State of Data Brzil 2021 - Figura 6, a experiência prévia
para cargos de maior senioridade é muito significativa, principalmente
em cargos sênior e de gestão, logo, isso pode estar sendo um
dificultador para elas alcançarem os cargos mais importantes. Tudo isso
se deve pelo fato das áreas de TI terem sido, durante muito tempo, quase
que exclusivamente destinadas aos homens (e ainda são, em certa medida).
Um dado que mostra isso é o da pesquisa realizada pelo Cadastro Geral de
Empregados e Desempregados (Caged), que aponta que a participação
feminina na área cresceu 60\% em 5 anos, passando de 27,9 mil mulheres,
em 2014, para 44,5 mil, em 2019, no entanto, esse número ainda
representa 20\% dos profissionais de TI atuantes no mercado(
\url{https://www.gazetadopovo.com.br/vozes/gazzconecta-colab/mulheres-tecnologia-como-impulsionar-presenca-feminina-setor/}
).

\begin{Shaded}
\begin{Highlighting}[]
\CommentTok{\#experiência por gênero}
\NormalTok{plot1}\OtherTok{=}\NormalTok{df}\SpecialCharTok{\%\textgreater{}\%}
\NormalTok{  dplyr}\SpecialCharTok{::}\FunctionTok{count}\NormalTok{(}\StringTok{\textasciigrave{}}\AttributeTok{(\textquotesingle{}P1\_b \textquotesingle{}, \textquotesingle{}Genero\textquotesingle{})}\StringTok{\textasciigrave{}}\NormalTok{, }\StringTok{\textasciigrave{}}\AttributeTok{Experiência na área de dados}\StringTok{\textasciigrave{}}\NormalTok{) }\SpecialCharTok{\%\textgreater{}\%}
\NormalTok{  dplyr}\SpecialCharTok{::}\FunctionTok{group\_by}\NormalTok{(}\StringTok{\textasciigrave{}}\AttributeTok{(\textquotesingle{}P1\_b \textquotesingle{}, \textquotesingle{}Genero\textquotesingle{})}\StringTok{\textasciigrave{}}\NormalTok{) }\SpecialCharTok{\%\textgreater{}\%}
\NormalTok{  dplyr}\SpecialCharTok{::}\FunctionTok{mutate}\NormalTok{(}\AttributeTok{Prop =}\NormalTok{ n}\SpecialCharTok{/}\FunctionTok{sum}\NormalTok{(n))}\SpecialCharTok{\%\textgreater{}\%}
\NormalTok{  ggplot2}\SpecialCharTok{::}\FunctionTok{ggplot}\NormalTok{(}
\NormalTok{    ggplot2}\SpecialCharTok{::}\FunctionTok{aes}\NormalTok{(}\AttributeTok{x =} \StringTok{\textasciigrave{}}\AttributeTok{(\textquotesingle{}P1\_b \textquotesingle{}, \textquotesingle{}Genero\textquotesingle{})}\StringTok{\textasciigrave{}}\NormalTok{, }\AttributeTok{y =}\NormalTok{ Prop,}
                 \AttributeTok{fill =} \StringTok{\textasciigrave{}}\AttributeTok{Experiência na área de dados}\StringTok{\textasciigrave{}}\NormalTok{)) }\SpecialCharTok{+}
\NormalTok{  ggplot2}\SpecialCharTok{::}\FunctionTok{geom\_col}\NormalTok{( }\AttributeTok{color=}\StringTok{"white"}\NormalTok{,}
                     \AttributeTok{position =}\NormalTok{ ggplot2}\SpecialCharTok{::}\FunctionTok{position\_fill}\NormalTok{()) }\SpecialCharTok{+}
\NormalTok{  ggrepel}\SpecialCharTok{::}\FunctionTok{geom\_label\_repel}\NormalTok{(ggplot2}\SpecialCharTok{::}\FunctionTok{aes}\NormalTok{(}
    \AttributeTok{label =}\NormalTok{ scales}\SpecialCharTok{::}\FunctionTok{percent}\NormalTok{(Prop)),}
    \AttributeTok{fontface =} \StringTok{\textquotesingle{}bold\textquotesingle{}}\NormalTok{,}
    \AttributeTok{hjust=}\DecValTok{2}\NormalTok{,}
    \AttributeTok{position =}\NormalTok{ ggplot2}\SpecialCharTok{::}\FunctionTok{position\_stack}\NormalTok{(}\AttributeTok{vjust =}\NormalTok{ .}\DecValTok{5}\NormalTok{),}
    \AttributeTok{size=}\FloatTok{3.5}\NormalTok{) }\SpecialCharTok{+}
\NormalTok{  ggplot2}\SpecialCharTok{::}\FunctionTok{ggtitle}\NormalTok{(}\StringTok{\textquotesingle{}Figura 7: Experiência na área de dados x Gênero\textquotesingle{}}\NormalTok{)}\SpecialCharTok{+}
\NormalTok{  ggplot2}\SpecialCharTok{::}\FunctionTok{theme\_void}\NormalTok{()}\SpecialCharTok{+}
\NormalTok{  ggplot2}\SpecialCharTok{::}\FunctionTok{scale\_fill\_brewer}\NormalTok{(}\AttributeTok{type =} \StringTok{"seq"}\NormalTok{, }\AttributeTok{palette =} \StringTok{"RdPu"}\NormalTok{)}\SpecialCharTok{+}
\NormalTok{  ggplot2}\SpecialCharTok{::}\FunctionTok{xlab}\NormalTok{(}\StringTok{\textquotesingle{}Gênero\textquotesingle{}}\NormalTok{)}\SpecialCharTok{+}
\NormalTok{  ggplot2}\SpecialCharTok{::}\FunctionTok{theme}\NormalTok{(}\AttributeTok{legend.position=}\StringTok{"bottom"}\NormalTok{,}
                 \AttributeTok{plot.title=}\NormalTok{ggplot2}\SpecialCharTok{::}\FunctionTok{element\_text}\NormalTok{(}\AttributeTok{face=}\StringTok{\textquotesingle{}bold.italic\textquotesingle{}}\NormalTok{,}
                                                  \AttributeTok{hjust =} \FloatTok{0.5}\NormalTok{, }\AttributeTok{size=}\DecValTok{20}\NormalTok{),}
                 \AttributeTok{axis.text.y=}\NormalTok{ggplot2}\SpecialCharTok{::}\FunctionTok{element\_blank}\NormalTok{(),}
                 \AttributeTok{axis.title.y=}\NormalTok{ggplot2}\SpecialCharTok{::}\FunctionTok{element\_blank}\NormalTok{(),}
                 \AttributeTok{axis.title.x=}\NormalTok{ggplot2}\SpecialCharTok{::}\FunctionTok{element\_blank}\NormalTok{(),}
                 \AttributeTok{axis.text.x =}\NormalTok{ggplot2}\SpecialCharTok{::}\FunctionTok{element\_text}\NormalTok{(}\AttributeTok{face=}\StringTok{\textquotesingle{}bold\textquotesingle{}}\NormalTok{, }\AttributeTok{size=}\DecValTok{12}\NormalTok{),}
                 \AttributeTok{legend.title=}\NormalTok{ggplot2}\SpecialCharTok{::}\FunctionTok{element\_blank}\NormalTok{())}

\NormalTok{plot2}\OtherTok{=}\NormalTok{df}\SpecialCharTok{\%\textgreater{}\%}
\NormalTok{  dplyr}\SpecialCharTok{::}\FunctionTok{count}\NormalTok{(}\StringTok{\textasciigrave{}}\AttributeTok{(\textquotesingle{}P1\_b \textquotesingle{}, \textquotesingle{}Genero\textquotesingle{})}\StringTok{\textasciigrave{}}\NormalTok{, }\StringTok{\textasciigrave{}}\AttributeTok{Experiência na área de TI}\StringTok{\textasciigrave{}}\NormalTok{) }\SpecialCharTok{\%\textgreater{}\%}
\NormalTok{  dplyr}\SpecialCharTok{::}\FunctionTok{group\_by}\NormalTok{(}\StringTok{\textasciigrave{}}\AttributeTok{(\textquotesingle{}P1\_b \textquotesingle{}, \textquotesingle{}Genero\textquotesingle{})}\StringTok{\textasciigrave{}}\NormalTok{) }\SpecialCharTok{\%\textgreater{}\%}
\NormalTok{  dplyr}\SpecialCharTok{::}\FunctionTok{mutate}\NormalTok{(}\AttributeTok{Prop =}\NormalTok{ n}\SpecialCharTok{/}\FunctionTok{sum}\NormalTok{(n))}\SpecialCharTok{\%\textgreater{}\%}
\NormalTok{  ggplot2}\SpecialCharTok{::}\FunctionTok{ggplot}\NormalTok{(}
\NormalTok{    ggplot2}\SpecialCharTok{::}\FunctionTok{aes}\NormalTok{(}\AttributeTok{x =} \StringTok{\textasciigrave{}}\AttributeTok{(\textquotesingle{}P1\_b \textquotesingle{}, \textquotesingle{}Genero\textquotesingle{})}\StringTok{\textasciigrave{}}\NormalTok{, }\AttributeTok{y =}\NormalTok{ Prop,}
                 \AttributeTok{fill =} \StringTok{\textasciigrave{}}\AttributeTok{Experiência na área de TI}\StringTok{\textasciigrave{}}\NormalTok{)) }\SpecialCharTok{+}
\NormalTok{  ggplot2}\SpecialCharTok{::}\FunctionTok{geom\_col}\NormalTok{( }\AttributeTok{color=}\StringTok{"white"}\NormalTok{,}
                     \AttributeTok{position =}\NormalTok{ ggplot2}\SpecialCharTok{::}\FunctionTok{position\_fill}\NormalTok{()) }\SpecialCharTok{+}
\NormalTok{  ggrepel}\SpecialCharTok{::}\FunctionTok{geom\_label\_repel}\NormalTok{(ggplot2}\SpecialCharTok{::}\FunctionTok{aes}\NormalTok{(}
    \AttributeTok{label =}\NormalTok{ scales}\SpecialCharTok{::}\FunctionTok{percent}\NormalTok{(Prop)),}
    \AttributeTok{fontface =} \StringTok{\textquotesingle{}bold\textquotesingle{}}\NormalTok{,}
    \AttributeTok{hjust=}\DecValTok{2}\NormalTok{,}
    \AttributeTok{position =}\NormalTok{ ggplot2}\SpecialCharTok{::}\FunctionTok{position\_stack}\NormalTok{(}\AttributeTok{vjust =}\NormalTok{ .}\DecValTok{5}\NormalTok{),}
    \AttributeTok{size=}\FloatTok{3.5}\NormalTok{) }\SpecialCharTok{+}
\NormalTok{  ggplot2}\SpecialCharTok{::}\FunctionTok{ggtitle}\NormalTok{(}\StringTok{\textquotesingle{}Figura 8: Experiência na área de TI x Gênero\textquotesingle{}}\NormalTok{)}\SpecialCharTok{+}
\NormalTok{  ggplot2}\SpecialCharTok{::}\FunctionTok{theme\_void}\NormalTok{()}\SpecialCharTok{+}
\NormalTok{  ggplot2}\SpecialCharTok{::}\FunctionTok{scale\_fill\_brewer}\NormalTok{(}\AttributeTok{type =} \StringTok{"seq"}\NormalTok{, }\AttributeTok{palette =} \StringTok{"PuRd"}\NormalTok{)}\SpecialCharTok{+}
\NormalTok{  ggplot2}\SpecialCharTok{::}\FunctionTok{xlab}\NormalTok{(}\StringTok{\textquotesingle{}Gênero\textquotesingle{}}\NormalTok{)}\SpecialCharTok{+}
\NormalTok{  ggplot2}\SpecialCharTok{::}\FunctionTok{theme}\NormalTok{(}\AttributeTok{legend.position=}\StringTok{"bottom"}\NormalTok{,}
                 \AttributeTok{plot.title=}\NormalTok{ggplot2}\SpecialCharTok{::}\FunctionTok{element\_text}\NormalTok{(}\AttributeTok{face=}\StringTok{\textquotesingle{}bold.italic\textquotesingle{}}\NormalTok{,}
                                                  \AttributeTok{hjust =} \FloatTok{0.5}\NormalTok{, }\AttributeTok{size=}\DecValTok{20}\NormalTok{),}
                 \AttributeTok{axis.text.y=}\NormalTok{ggplot2}\SpecialCharTok{::}\FunctionTok{element\_blank}\NormalTok{(),}
                 \AttributeTok{axis.title.y=}\NormalTok{ggplot2}\SpecialCharTok{::}\FunctionTok{element\_blank}\NormalTok{(),}
                 \AttributeTok{axis.title.x=}\NormalTok{ggplot2}\SpecialCharTok{::}\FunctionTok{element\_blank}\NormalTok{(),}
                 \AttributeTok{axis.text.x =}\NormalTok{ggplot2}\SpecialCharTok{::}\FunctionTok{element\_text}\NormalTok{(}\AttributeTok{face=}\StringTok{\textquotesingle{}bold\textquotesingle{}}\NormalTok{, }\AttributeTok{size=}\DecValTok{12}\NormalTok{),}
                 \AttributeTok{legend.title=}\NormalTok{ggplot2}\SpecialCharTok{::}\FunctionTok{element\_blank}\NormalTok{())}

\NormalTok{gridExtra}\SpecialCharTok{::}\FunctionTok{grid.arrange}\NormalTok{(plot1, plot2)}
\end{Highlighting}
\end{Shaded}

\includegraphics{relatorio_files/figure-latex/unnamed-chunk-6-1.pdf}

Quando analisamos a experiência por nível de cargo, percebe-se, de uma
maneira geral, um equilibrio, principalmente nos cargos pleno e sênior,
além de uma leve vantagem para os homens entrevistados nos cargos júnior
e de gestor. Em relação à experiência prévia em TI, a vantagem entre os
entrevistados é, como esperado, bem maior. 14\% das mulheres Sênior
disseram ter 4 ou mais anos de experiência contra 36,4\% dos homens
sênior, enquanto que, mais de 51\% delas não tiveram experiência contra
33\% deles. Entre os gestores, o ``placar'' é de 26,2\% x 35,3\% para os
homens entre aqueles que possuem 4 ou mais anos de experiência prévia e
41\% x 32,3\% para as mulheres entre os sem experiência em TI, sendo
que, cargos de gestão exigem bastante experiência (o que pode explicar,
em partes, a maior remuneração dos gestores homens).

\begin{Shaded}
\begin{Highlighting}[]
\CommentTok{\#Experiência por gênero e nível de cargo}
\NormalTok{df}\SpecialCharTok{\%\textgreater{}\%}
\NormalTok{  dplyr}\SpecialCharTok{::}\FunctionTok{count}\NormalTok{(}\StringTok{\textasciigrave{}}\AttributeTok{(\textquotesingle{}P2\_g \textquotesingle{}, \textquotesingle{}Nivel\textquotesingle{})}\StringTok{\textasciigrave{}}\NormalTok{,}\StringTok{\textasciigrave{}}\AttributeTok{(\textquotesingle{}P1\_b \textquotesingle{}, \textquotesingle{}Genero\textquotesingle{})}\StringTok{\textasciigrave{}}\NormalTok{,}
               \StringTok{\textasciigrave{}}\AttributeTok{Experiência na área de dados}\StringTok{\textasciigrave{}}\NormalTok{) }\SpecialCharTok{\%\textgreater{}\%}
\NormalTok{  dplyr}\SpecialCharTok{::}\FunctionTok{group\_by}\NormalTok{(}\StringTok{\textasciigrave{}}\AttributeTok{(\textquotesingle{}P2\_g \textquotesingle{}, \textquotesingle{}Nivel\textquotesingle{})}\StringTok{\textasciigrave{}}\NormalTok{,}\StringTok{\textasciigrave{}}\AttributeTok{(\textquotesingle{}P1\_b \textquotesingle{}, \textquotesingle{}Genero\textquotesingle{})}\StringTok{\textasciigrave{}}\NormalTok{) }\SpecialCharTok{\%\textgreater{}\%}
\NormalTok{  dplyr}\SpecialCharTok{::}\FunctionTok{mutate}\NormalTok{(}\AttributeTok{Prop =}\NormalTok{ n}\SpecialCharTok{/}\FunctionTok{sum}\NormalTok{(n))}\SpecialCharTok{\%\textgreater{}\%}
\NormalTok{  ggplot2}\SpecialCharTok{::}\FunctionTok{ggplot}\NormalTok{(}
\NormalTok{    ggplot2}\SpecialCharTok{::}\FunctionTok{aes}\NormalTok{(}\AttributeTok{x =} \StringTok{\textasciigrave{}}\AttributeTok{(\textquotesingle{}P1\_b \textquotesingle{}, \textquotesingle{}Genero\textquotesingle{})}\StringTok{\textasciigrave{}}\NormalTok{, }\AttributeTok{y =}\NormalTok{ Prop,}
                 \AttributeTok{fill =} \StringTok{\textasciigrave{}}\AttributeTok{Experiência na área de dados}\StringTok{\textasciigrave{}}\NormalTok{)) }\SpecialCharTok{+}
\NormalTok{  ggplot2}\SpecialCharTok{::}\FunctionTok{geom\_col}\NormalTok{( }\AttributeTok{color=}\StringTok{"white"}\NormalTok{,}
                     \AttributeTok{position =}\NormalTok{ ggplot2}\SpecialCharTok{::}\FunctionTok{position\_fill}\NormalTok{()) }\SpecialCharTok{+}
\NormalTok{  ggrepel}\SpecialCharTok{::}\FunctionTok{geom\_label\_repel}\NormalTok{(ggplot2}\SpecialCharTok{::}\FunctionTok{aes}\NormalTok{(}
    \AttributeTok{label =}\NormalTok{ scales}\SpecialCharTok{::}\FunctionTok{percent}\NormalTok{(Prop)),}
    \AttributeTok{fontface =} \StringTok{\textquotesingle{}bold\textquotesingle{}}\NormalTok{,}
    \AttributeTok{hjust=}\DecValTok{2}\NormalTok{,}
    \AttributeTok{position =}\NormalTok{ ggplot2}\SpecialCharTok{::}\FunctionTok{position\_stack}\NormalTok{(}\AttributeTok{vjust =}\NormalTok{ .}\DecValTok{5}\NormalTok{),}
    \AttributeTok{size=}\FloatTok{3.5}\NormalTok{) }\SpecialCharTok{+}
\NormalTok{  ggplot2}\SpecialCharTok{::}\FunctionTok{ggtitle}\NormalTok{(}\StringTok{\textquotesingle{}Figura 9: Experiência na área de dados x Nível de Cargo x Gênero\textquotesingle{}}\NormalTok{)}\SpecialCharTok{+}
\NormalTok{  ggplot2}\SpecialCharTok{::}\FunctionTok{theme\_void}\NormalTok{()}\SpecialCharTok{+}
\NormalTok{  ggplot2}\SpecialCharTok{::}\FunctionTok{scale\_fill\_brewer}\NormalTok{(}\AttributeTok{type =} \StringTok{"seq"}\NormalTok{, }\AttributeTok{palette =} \StringTok{"RdPu"}\NormalTok{)}\SpecialCharTok{+}
\NormalTok{  ggplot2}\SpecialCharTok{::}\FunctionTok{xlab}\NormalTok{(}\StringTok{\textquotesingle{}Gênero\textquotesingle{}}\NormalTok{)}\SpecialCharTok{+}
\NormalTok{  ggplot2}\SpecialCharTok{::}\FunctionTok{theme}\NormalTok{(}\AttributeTok{legend.position=}\StringTok{"bottom"}\NormalTok{,}
                 \AttributeTok{plot.title=}\NormalTok{ggplot2}\SpecialCharTok{::}\FunctionTok{element\_text}\NormalTok{(}\AttributeTok{face=}\StringTok{\textquotesingle{}bold.italic\textquotesingle{}}\NormalTok{,}
                                                  \AttributeTok{hjust =} \FloatTok{0.5}\NormalTok{, }\AttributeTok{size=}\DecValTok{20}\NormalTok{),}
                 \AttributeTok{axis.text.y=}\NormalTok{ggplot2}\SpecialCharTok{::}\FunctionTok{element\_blank}\NormalTok{(),}
                 \AttributeTok{axis.title.y=}\NormalTok{ggplot2}\SpecialCharTok{::}\FunctionTok{element\_blank}\NormalTok{(),}
                 \AttributeTok{axis.title.x=}\NormalTok{ggplot2}\SpecialCharTok{::}\FunctionTok{element\_blank}\NormalTok{(),}
                 \AttributeTok{axis.text.x =}\NormalTok{ggplot2}\SpecialCharTok{::}\FunctionTok{element\_text}\NormalTok{(}\AttributeTok{face=}\StringTok{\textquotesingle{}bold\textquotesingle{}}\NormalTok{, }\AttributeTok{size=}\DecValTok{12}\NormalTok{),}
                 \AttributeTok{legend.title=}\NormalTok{ggplot2}\SpecialCharTok{::}\FunctionTok{element\_blank}\NormalTok{())}\SpecialCharTok{+}
\NormalTok{  ggplot2}\SpecialCharTok{::}\FunctionTok{facet\_grid}\NormalTok{(}\SpecialCharTok{\textasciitilde{}}\StringTok{\textasciigrave{}}\AttributeTok{(\textquotesingle{}P2\_g \textquotesingle{}, \textquotesingle{}Nivel\textquotesingle{})}\StringTok{\textasciigrave{}}\NormalTok{)}
\end{Highlighting}
\end{Shaded}

\includegraphics{relatorio_files/figure-latex/unnamed-chunk-7-1.pdf}

\begin{Shaded}
\begin{Highlighting}[]
\NormalTok{df}\SpecialCharTok{\%\textgreater{}\%}
\NormalTok{  dplyr}\SpecialCharTok{::}\FunctionTok{count}\NormalTok{(}\StringTok{\textasciigrave{}}\AttributeTok{(\textquotesingle{}P2\_g \textquotesingle{}, \textquotesingle{}Nivel\textquotesingle{})}\StringTok{\textasciigrave{}}\NormalTok{,}\StringTok{\textasciigrave{}}\AttributeTok{(\textquotesingle{}P1\_b \textquotesingle{}, \textquotesingle{}Genero\textquotesingle{})}\StringTok{\textasciigrave{}}\NormalTok{, }\StringTok{\textasciigrave{}}\AttributeTok{Experiência na área de TI}\StringTok{\textasciigrave{}}\NormalTok{) }\SpecialCharTok{\%\textgreater{}\%}
\NormalTok{  dplyr}\SpecialCharTok{::}\FunctionTok{group\_by}\NormalTok{(}\StringTok{\textasciigrave{}}\AttributeTok{(\textquotesingle{}P2\_g \textquotesingle{}, \textquotesingle{}Nivel\textquotesingle{})}\StringTok{\textasciigrave{}}\NormalTok{,}\StringTok{\textasciigrave{}}\AttributeTok{(\textquotesingle{}P1\_b \textquotesingle{}, \textquotesingle{}Genero\textquotesingle{})}\StringTok{\textasciigrave{}}\NormalTok{) }\SpecialCharTok{\%\textgreater{}\%}
\NormalTok{  dplyr}\SpecialCharTok{::}\FunctionTok{mutate}\NormalTok{(}\AttributeTok{Prop =}\NormalTok{ n}\SpecialCharTok{/}\FunctionTok{sum}\NormalTok{(n))}\SpecialCharTok{\%\textgreater{}\%}
\NormalTok{  ggplot2}\SpecialCharTok{::}\FunctionTok{ggplot}\NormalTok{(}
\NormalTok{    ggplot2}\SpecialCharTok{::}\FunctionTok{aes}\NormalTok{(}\AttributeTok{x =} \StringTok{\textasciigrave{}}\AttributeTok{(\textquotesingle{}P1\_b \textquotesingle{}, \textquotesingle{}Genero\textquotesingle{})}\StringTok{\textasciigrave{}}\NormalTok{, }\AttributeTok{y =}\NormalTok{ Prop,}
                 \AttributeTok{fill =} \StringTok{\textasciigrave{}}\AttributeTok{Experiência na área de TI}\StringTok{\textasciigrave{}}\NormalTok{)) }\SpecialCharTok{+}
\NormalTok{  ggplot2}\SpecialCharTok{::}\FunctionTok{geom\_col}\NormalTok{( }\AttributeTok{color=}\StringTok{"white"}\NormalTok{,}
                     \AttributeTok{position =}\NormalTok{ ggplot2}\SpecialCharTok{::}\FunctionTok{position\_fill}\NormalTok{()) }\SpecialCharTok{+}
\NormalTok{  ggrepel}\SpecialCharTok{::}\FunctionTok{geom\_label\_repel}\NormalTok{(ggplot2}\SpecialCharTok{::}\FunctionTok{aes}\NormalTok{(}
    \AttributeTok{label =}\NormalTok{ scales}\SpecialCharTok{::}\FunctionTok{percent}\NormalTok{(Prop)),}
    \AttributeTok{fontface =} \StringTok{\textquotesingle{}bold\textquotesingle{}}\NormalTok{,}
    \AttributeTok{hjust=}\DecValTok{2}\NormalTok{,}
    \AttributeTok{position =}\NormalTok{ ggplot2}\SpecialCharTok{::}\FunctionTok{position\_stack}\NormalTok{(}\AttributeTok{vjust =}\NormalTok{ .}\DecValTok{5}\NormalTok{),}
    \AttributeTok{size=}\FloatTok{3.5}\NormalTok{) }\SpecialCharTok{+}
\NormalTok{  ggplot2}\SpecialCharTok{::}\FunctionTok{ggtitle}\NormalTok{(}\StringTok{\textquotesingle{}Figura 10: Experiência na área de TI x Nível de Cargo x Gênero\textquotesingle{}}\NormalTok{)}\SpecialCharTok{+}
\NormalTok{  ggplot2}\SpecialCharTok{::}\FunctionTok{theme\_void}\NormalTok{()}\SpecialCharTok{+}
\NormalTok{  ggplot2}\SpecialCharTok{::}\FunctionTok{scale\_fill\_brewer}\NormalTok{(}\AttributeTok{type =} \StringTok{"seq"}\NormalTok{, }\AttributeTok{palette =} \StringTok{"PuRd"}\NormalTok{)}\SpecialCharTok{+}
\NormalTok{  ggplot2}\SpecialCharTok{::}\FunctionTok{xlab}\NormalTok{(}\StringTok{\textquotesingle{}Gênero\textquotesingle{}}\NormalTok{)}\SpecialCharTok{+}
\NormalTok{  ggplot2}\SpecialCharTok{::}\FunctionTok{theme}\NormalTok{(}\AttributeTok{legend.position=}\StringTok{"bottom"}\NormalTok{,}
                 \AttributeTok{plot.title=}\NormalTok{ggplot2}\SpecialCharTok{::}\FunctionTok{element\_text}\NormalTok{(}\AttributeTok{face=}\StringTok{\textquotesingle{}bold.italic\textquotesingle{}}\NormalTok{,}
                                                  \AttributeTok{hjust =} \FloatTok{0.5}\NormalTok{, }\AttributeTok{size=}\DecValTok{20}\NormalTok{),}
                 \AttributeTok{axis.text.y=}\NormalTok{ggplot2}\SpecialCharTok{::}\FunctionTok{element\_blank}\NormalTok{(),}
                 \AttributeTok{axis.title.y=}\NormalTok{ggplot2}\SpecialCharTok{::}\FunctionTok{element\_blank}\NormalTok{(),}
                 \AttributeTok{axis.title.x=}\NormalTok{ggplot2}\SpecialCharTok{::}\FunctionTok{element\_blank}\NormalTok{(),}
                 \AttributeTok{axis.text.x =}\NormalTok{ggplot2}\SpecialCharTok{::}\FunctionTok{element\_text}\NormalTok{(}\AttributeTok{face=}\StringTok{\textquotesingle{}bold\textquotesingle{}}\NormalTok{, }\AttributeTok{size=}\DecValTok{12}\NormalTok{),}
                 \AttributeTok{legend.title=}\NormalTok{ggplot2}\SpecialCharTok{::}\FunctionTok{element\_blank}\NormalTok{())}\SpecialCharTok{+}
\NormalTok{  ggplot2}\SpecialCharTok{::}\FunctionTok{facet\_grid}\NormalTok{(}\SpecialCharTok{\textasciitilde{}}\StringTok{\textasciigrave{}}\AttributeTok{(\textquotesingle{}P2\_g \textquotesingle{}, \textquotesingle{}Nivel\textquotesingle{})}\StringTok{\textasciigrave{}}\NormalTok{)}
\end{Highlighting}
\end{Shaded}

\includegraphics{relatorio_files/figure-latex/unnamed-chunk-7-2.pdf}

Entre os analistas e, principalmente, cientistas de dados, há equilíbrio
entre os gêneros em termos de experiência na área. Já entre os
engenheiros de dados, a vantagem é ampla dos homens -já que faltam
engenheiras de dados sênior, que seriam as profissionais mais
experientes (Figura 2) -, sendo 26,79\% das mulheres possuindo o mínimo
de 4 anos de vivência na área, enquanto que os homens entrevistados
aparecem com 41,3\%. Considerando a experiência anterior em tI, em todas
as funções, a vantagem, entre os entrevistados, é masculina, sendo que,
entre os engenheiros de dados, 10,67\% das mulheres possuem 6 ou mais
anos de experiência, já que faltam engenheiras sênior enquanto que,
entre os homens, este percentual é de mais de 25\%. Aparentemente, por
se tratar de uma função extremamente técnica, o que exigiría mais
experiência em tI, há poucas engenheiras sênior. Além disso, a função de
análise de dados exige menos experiência em TI, o que pode explicar a
maior ``adesão'' relativa feminina.

\begin{Shaded}
\begin{Highlighting}[]
\CommentTok{\#Experiência por gênero e atuação}
\NormalTok{df}\SpecialCharTok{\%\textgreater{}\%}
\NormalTok{  dplyr}\SpecialCharTok{::}\FunctionTok{count}\NormalTok{(}\StringTok{\textasciigrave{}}\AttributeTok{(\textquotesingle{}P4\_a \textquotesingle{}, \textquotesingle{}Atuacao\textquotesingle{})}\StringTok{\textasciigrave{}}\NormalTok{,}\StringTok{\textasciigrave{}}\AttributeTok{(\textquotesingle{}P1\_b \textquotesingle{}, \textquotesingle{}Genero\textquotesingle{})}\StringTok{\textasciigrave{}}\NormalTok{,}
               \StringTok{\textasciigrave{}}\AttributeTok{Experiência na área de dados}\StringTok{\textasciigrave{}}\NormalTok{) }\SpecialCharTok{\%\textgreater{}\%}
\NormalTok{  dplyr}\SpecialCharTok{::}\FunctionTok{group\_by}\NormalTok{(}\StringTok{\textasciigrave{}}\AttributeTok{(\textquotesingle{}P4\_a \textquotesingle{}, \textquotesingle{}Atuacao\textquotesingle{})}\StringTok{\textasciigrave{}}\NormalTok{,}\StringTok{\textasciigrave{}}\AttributeTok{(\textquotesingle{}P1\_b \textquotesingle{}, \textquotesingle{}Genero\textquotesingle{})}\StringTok{\textasciigrave{}}\NormalTok{) }\SpecialCharTok{\%\textgreater{}\%}
\NormalTok{  dplyr}\SpecialCharTok{::}\FunctionTok{mutate}\NormalTok{(}\AttributeTok{Prop =}\NormalTok{ n}\SpecialCharTok{/}\FunctionTok{sum}\NormalTok{(n))}\SpecialCharTok{\%\textgreater{}\%}
\NormalTok{  dplyr}\SpecialCharTok{::}\FunctionTok{filter}\NormalTok{(}\StringTok{\textasciigrave{}}\AttributeTok{(\textquotesingle{}P4\_a \textquotesingle{}, \textquotesingle{}Atuacao\textquotesingle{})}\StringTok{\textasciigrave{}}\SpecialCharTok{\%notin\%}\FunctionTok{c}\NormalTok{(}\StringTok{\textquotesingle{}Outra\textquotesingle{}}\NormalTok{,}\StringTok{\textquotesingle{}Gestor\textquotesingle{}}\NormalTok{))}\SpecialCharTok{\%\textgreater{}\%}
\NormalTok{  ggplot2}\SpecialCharTok{::}\FunctionTok{ggplot}\NormalTok{(}
\NormalTok{    ggplot2}\SpecialCharTok{::}\FunctionTok{aes}\NormalTok{(}\AttributeTok{x =} \StringTok{\textasciigrave{}}\AttributeTok{(\textquotesingle{}P1\_b \textquotesingle{}, \textquotesingle{}Genero\textquotesingle{})}\StringTok{\textasciigrave{}}\NormalTok{, }\AttributeTok{y =}\NormalTok{ Prop,}
                 \AttributeTok{fill =} \StringTok{\textasciigrave{}}\AttributeTok{Experiência na área de dados}\StringTok{\textasciigrave{}}\NormalTok{)) }\SpecialCharTok{+}
\NormalTok{  ggplot2}\SpecialCharTok{::}\FunctionTok{geom\_col}\NormalTok{( }\AttributeTok{color=}\StringTok{"white"}\NormalTok{,}
                     \AttributeTok{position =}\NormalTok{ ggplot2}\SpecialCharTok{::}\FunctionTok{position\_fill}\NormalTok{()) }\SpecialCharTok{+}
\NormalTok{  ggrepel}\SpecialCharTok{::}\FunctionTok{geom\_label\_repel}\NormalTok{(ggplot2}\SpecialCharTok{::}\FunctionTok{aes}\NormalTok{(}
    \AttributeTok{label =}\NormalTok{ scales}\SpecialCharTok{::}\FunctionTok{percent}\NormalTok{(Prop)),}
    \AttributeTok{fontface =} \StringTok{\textquotesingle{}bold\textquotesingle{}}\NormalTok{,}
    \AttributeTok{hjust=}\DecValTok{2}\NormalTok{,}
    \AttributeTok{position =}\NormalTok{ ggplot2}\SpecialCharTok{::}\FunctionTok{position\_stack}\NormalTok{(}\AttributeTok{vjust =}\NormalTok{ .}\DecValTok{5}\NormalTok{),}
    \AttributeTok{size=}\FloatTok{3.5}\NormalTok{) }\SpecialCharTok{+}
\NormalTok{  ggplot2}\SpecialCharTok{::}\FunctionTok{ggtitle}\NormalTok{(}\StringTok{\textquotesingle{}Figura 11: Experiência na área de dados x Atuação x Gênero\textquotesingle{}}\NormalTok{)}\SpecialCharTok{+}
\NormalTok{  ggplot2}\SpecialCharTok{::}\FunctionTok{theme\_void}\NormalTok{()}\SpecialCharTok{+}
\NormalTok{  ggplot2}\SpecialCharTok{::}\FunctionTok{scale\_fill\_brewer}\NormalTok{(}\AttributeTok{type =} \StringTok{"seq"}\NormalTok{, }\AttributeTok{palette =} \StringTok{"RdPu"}\NormalTok{)}\SpecialCharTok{+}
\NormalTok{  ggplot2}\SpecialCharTok{::}\FunctionTok{xlab}\NormalTok{(}\StringTok{\textquotesingle{}Gênero\textquotesingle{}}\NormalTok{)}\SpecialCharTok{+}
\NormalTok{  ggplot2}\SpecialCharTok{::}\FunctionTok{theme}\NormalTok{(}\AttributeTok{legend.position=}\StringTok{"bottom"}\NormalTok{,}
                 \AttributeTok{plot.title=}\NormalTok{ggplot2}\SpecialCharTok{::}\FunctionTok{element\_text}\NormalTok{(}\AttributeTok{face=}\StringTok{\textquotesingle{}bold.italic\textquotesingle{}}\NormalTok{,}
                                                  \AttributeTok{hjust =} \FloatTok{0.5}\NormalTok{, }\AttributeTok{size=}\DecValTok{20}\NormalTok{),}
                 \AttributeTok{axis.text.y=}\NormalTok{ggplot2}\SpecialCharTok{::}\FunctionTok{element\_blank}\NormalTok{(),}
                 \AttributeTok{axis.title.y=}\NormalTok{ggplot2}\SpecialCharTok{::}\FunctionTok{element\_blank}\NormalTok{(),}
                 \AttributeTok{axis.title.x=}\NormalTok{ggplot2}\SpecialCharTok{::}\FunctionTok{element\_blank}\NormalTok{(),}
                 \AttributeTok{axis.text.x =}\NormalTok{ggplot2}\SpecialCharTok{::}\FunctionTok{element\_text}\NormalTok{(}\AttributeTok{face=}\StringTok{\textquotesingle{}bold\textquotesingle{}}\NormalTok{, }\AttributeTok{size=}\DecValTok{12}\NormalTok{),}
                 \AttributeTok{legend.title=}\NormalTok{ggplot2}\SpecialCharTok{::}\FunctionTok{element\_blank}\NormalTok{())}\SpecialCharTok{+}
\NormalTok{  ggplot2}\SpecialCharTok{::}\FunctionTok{facet\_grid}\NormalTok{(}\SpecialCharTok{\textasciitilde{}}\StringTok{\textasciigrave{}}\AttributeTok{(\textquotesingle{}P4\_a \textquotesingle{}, \textquotesingle{}Atuacao\textquotesingle{})}\StringTok{\textasciigrave{}}\NormalTok{)}
\end{Highlighting}
\end{Shaded}

\includegraphics{relatorio_files/figure-latex/unnamed-chunk-8-1.pdf}

\begin{Shaded}
\begin{Highlighting}[]
\NormalTok{df}\SpecialCharTok{\%\textgreater{}\%}
\NormalTok{  dplyr}\SpecialCharTok{::}\FunctionTok{count}\NormalTok{(}\StringTok{\textasciigrave{}}\AttributeTok{(\textquotesingle{}P4\_a \textquotesingle{}, \textquotesingle{}Atuacao\textquotesingle{})}\StringTok{\textasciigrave{}}\NormalTok{,}\StringTok{\textasciigrave{}}\AttributeTok{(\textquotesingle{}P1\_b \textquotesingle{}, \textquotesingle{}Genero\textquotesingle{})}\StringTok{\textasciigrave{}}\NormalTok{, }\StringTok{\textasciigrave{}}\AttributeTok{Experiência na área de TI}\StringTok{\textasciigrave{}}\NormalTok{) }\SpecialCharTok{\%\textgreater{}\%}
\NormalTok{  dplyr}\SpecialCharTok{::}\FunctionTok{group\_by}\NormalTok{(}\StringTok{\textasciigrave{}}\AttributeTok{(\textquotesingle{}P4\_a \textquotesingle{}, \textquotesingle{}Atuacao\textquotesingle{})}\StringTok{\textasciigrave{}}\NormalTok{,}\StringTok{\textasciigrave{}}\AttributeTok{(\textquotesingle{}P1\_b \textquotesingle{}, \textquotesingle{}Genero\textquotesingle{})}\StringTok{\textasciigrave{}}\NormalTok{) }\SpecialCharTok{\%\textgreater{}\%}
\NormalTok{  dplyr}\SpecialCharTok{::}\FunctionTok{mutate}\NormalTok{(}\AttributeTok{Prop =}\NormalTok{ n}\SpecialCharTok{/}\FunctionTok{sum}\NormalTok{(n))}\SpecialCharTok{\%\textgreater{}\%}
\NormalTok{  dplyr}\SpecialCharTok{::}\FunctionTok{filter}\NormalTok{(}\StringTok{\textasciigrave{}}\AttributeTok{(\textquotesingle{}P4\_a \textquotesingle{}, \textquotesingle{}Atuacao\textquotesingle{})}\StringTok{\textasciigrave{}}\SpecialCharTok{\%notin\%}\FunctionTok{c}\NormalTok{(}\StringTok{\textquotesingle{}Outra\textquotesingle{}}\NormalTok{,}\StringTok{\textquotesingle{}Gestor\textquotesingle{}}\NormalTok{))}\SpecialCharTok{\%\textgreater{}\%}
\NormalTok{  ggplot2}\SpecialCharTok{::}\FunctionTok{ggplot}\NormalTok{(}
\NormalTok{    ggplot2}\SpecialCharTok{::}\FunctionTok{aes}\NormalTok{(}\AttributeTok{x =} \StringTok{\textasciigrave{}}\AttributeTok{(\textquotesingle{}P1\_b \textquotesingle{}, \textquotesingle{}Genero\textquotesingle{})}\StringTok{\textasciigrave{}}\NormalTok{, }\AttributeTok{y =}\NormalTok{ Prop,}
                 \AttributeTok{fill =} \StringTok{\textasciigrave{}}\AttributeTok{Experiência na área de TI}\StringTok{\textasciigrave{}}\NormalTok{)) }\SpecialCharTok{+}
\NormalTok{  ggplot2}\SpecialCharTok{::}\FunctionTok{geom\_col}\NormalTok{( }\AttributeTok{color=}\StringTok{"white"}\NormalTok{,}
                     \AttributeTok{position =}\NormalTok{ ggplot2}\SpecialCharTok{::}\FunctionTok{position\_fill}\NormalTok{()) }\SpecialCharTok{+}
\NormalTok{  ggrepel}\SpecialCharTok{::}\FunctionTok{geom\_label\_repel}\NormalTok{(ggplot2}\SpecialCharTok{::}\FunctionTok{aes}\NormalTok{(}
    \AttributeTok{label =}\NormalTok{ scales}\SpecialCharTok{::}\FunctionTok{percent}\NormalTok{(Prop)),}
    \AttributeTok{fontface =} \StringTok{\textquotesingle{}bold\textquotesingle{}}\NormalTok{,}
    \AttributeTok{hjust=}\DecValTok{2}\NormalTok{,}
    \AttributeTok{position =}\NormalTok{ ggplot2}\SpecialCharTok{::}\FunctionTok{position\_stack}\NormalTok{(}\AttributeTok{vjust =}\NormalTok{ .}\DecValTok{5}\NormalTok{),}
    \AttributeTok{size=}\FloatTok{3.5}\NormalTok{) }\SpecialCharTok{+}
\NormalTok{  ggplot2}\SpecialCharTok{::}\FunctionTok{ggtitle}\NormalTok{(}\StringTok{\textquotesingle{}Figura 12: Experiência na área de TI x Atuação x Gênero\textquotesingle{}}\NormalTok{)}\SpecialCharTok{+}
\NormalTok{  ggplot2}\SpecialCharTok{::}\FunctionTok{theme\_void}\NormalTok{()}\SpecialCharTok{+}
\NormalTok{  ggplot2}\SpecialCharTok{::}\FunctionTok{scale\_fill\_brewer}\NormalTok{(}\AttributeTok{type =} \StringTok{"seq"}\NormalTok{, }\AttributeTok{palette =} \StringTok{"PuRd"}\NormalTok{)}\SpecialCharTok{+}
\NormalTok{  ggplot2}\SpecialCharTok{::}\FunctionTok{xlab}\NormalTok{(}\StringTok{\textquotesingle{}Gênero\textquotesingle{}}\NormalTok{)}\SpecialCharTok{+}
\NormalTok{  ggplot2}\SpecialCharTok{::}\FunctionTok{theme}\NormalTok{(}\AttributeTok{legend.position=}\StringTok{"bottom"}\NormalTok{,}
                 \AttributeTok{plot.title=}\NormalTok{ggplot2}\SpecialCharTok{::}\FunctionTok{element\_text}\NormalTok{(}\AttributeTok{face=}\StringTok{\textquotesingle{}bold.italic\textquotesingle{}}\NormalTok{,}
                                                  \AttributeTok{hjust =} \FloatTok{0.5}\NormalTok{, }\AttributeTok{size=}\DecValTok{20}\NormalTok{),}
                 \AttributeTok{axis.text.y=}\NormalTok{ggplot2}\SpecialCharTok{::}\FunctionTok{element\_blank}\NormalTok{(),}
                 \AttributeTok{axis.title.y=}\NormalTok{ggplot2}\SpecialCharTok{::}\FunctionTok{element\_blank}\NormalTok{(),}
                 \AttributeTok{axis.title.x=}\NormalTok{ggplot2}\SpecialCharTok{::}\FunctionTok{element\_blank}\NormalTok{(),}
                 \AttributeTok{axis.text.x =}\NormalTok{ggplot2}\SpecialCharTok{::}\FunctionTok{element\_text}\NormalTok{(}\AttributeTok{face=}\StringTok{\textquotesingle{}bold\textquotesingle{}}\NormalTok{, }\AttributeTok{size=}\DecValTok{12}\NormalTok{),}
                 \AttributeTok{legend.title=}\NormalTok{ggplot2}\SpecialCharTok{::}\FunctionTok{element\_blank}\NormalTok{())}\SpecialCharTok{+}
\NormalTok{  ggplot2}\SpecialCharTok{::}\FunctionTok{facet\_grid}\NormalTok{(}\SpecialCharTok{\textasciitilde{}}\StringTok{\textasciigrave{}}\AttributeTok{(\textquotesingle{}P4\_a \textquotesingle{}, \textquotesingle{}Atuacao\textquotesingle{})}\StringTok{\textasciigrave{}}\NormalTok{)}
\end{Highlighting}
\end{Shaded}

\includegraphics{relatorio_files/figure-latex/unnamed-chunk-8-2.pdf}

Analisando as faixas de idade em cada nível de cargo, podemos observar,
entre os cargos júnior, uma leve ``superioridade'' feminina, em relação
aos homens entrevistados, em termos de concentração nas faixas de mais
idade, o que pode ser mais um sinal de que a área de dados vem recebendo
mulheres vindas de outras áreas de atuação fora do campo da TI, assim
como a maior diversidade encontrada nos tipos de formação delas em
relação aos homens. Já em relação aos cargos pleno e sênior, os homens é
quem estão mais concentrados nas últimas faixas de idade. Entre os
gestores, no entanto, há equilibrio nas idades entre os 2 gêneros.

Olhando as faixas de idade por função, entre os engenheiros de dados, os
homens entrevistados ficaram bem mais concentrados nas idades mais
avançadas do que as mulheres, sendo que quase 49\% deles têm 35 anos ou
mais, enquanto que, entre as engenheiras de dados, este valor é cerca de
28,5\%. Isto reforça a tese de que há excassez de pessoas do gênero
feminino com bagagem o suficiente para assumir os cargos de sênior
(Figura 12), embora o nível de ensino das mulheres nesta função seja
maior que o dos homens.

\begin{Shaded}
\begin{Highlighting}[]
\CommentTok{\#Faixas de idade por gênero e nível de cargo}
\NormalTok{df}\SpecialCharTok{\%\textgreater{}\%}
\NormalTok{  dplyr}\SpecialCharTok{::}\FunctionTok{count}\NormalTok{(}\StringTok{\textasciigrave{}}\AttributeTok{(\textquotesingle{}P2\_g \textquotesingle{}, \textquotesingle{}Nivel\textquotesingle{})}\StringTok{\textasciigrave{}}\NormalTok{,}\StringTok{\textasciigrave{}}\AttributeTok{(\textquotesingle{}P1\_b \textquotesingle{}, \textquotesingle{}Genero\textquotesingle{})}\StringTok{\textasciigrave{}}\NormalTok{,}
               \StringTok{\textasciigrave{}}\AttributeTok{(\textquotesingle{}P1\_a\_a \textquotesingle{}, \textquotesingle{}Faixa idade\textquotesingle{})}\StringTok{\textasciigrave{}}\NormalTok{) }\SpecialCharTok{\%\textgreater{}\%}
\NormalTok{  dplyr}\SpecialCharTok{::}\FunctionTok{group\_by}\NormalTok{(}\StringTok{\textasciigrave{}}\AttributeTok{(\textquotesingle{}P2\_g \textquotesingle{}, \textquotesingle{}Nivel\textquotesingle{})}\StringTok{\textasciigrave{}}\NormalTok{,}\StringTok{\textasciigrave{}}\AttributeTok{(\textquotesingle{}P1\_b \textquotesingle{}, \textquotesingle{}Genero\textquotesingle{})}\StringTok{\textasciigrave{}}\NormalTok{) }\SpecialCharTok{\%\textgreater{}\%}
\NormalTok{  dplyr}\SpecialCharTok{::}\FunctionTok{mutate}\NormalTok{(}\AttributeTok{Prop =}\NormalTok{ n}\SpecialCharTok{/}\FunctionTok{sum}\NormalTok{(n))}\SpecialCharTok{\%\textgreater{}\%}
\NormalTok{  dplyr}\SpecialCharTok{::}\FunctionTok{filter}\NormalTok{(}\StringTok{\textasciigrave{}}\AttributeTok{(\textquotesingle{}P2\_g \textquotesingle{}, \textquotesingle{}Nivel\textquotesingle{})}\StringTok{\textasciigrave{}}\SpecialCharTok{\%notin\%}\FunctionTok{c}\NormalTok{(}\StringTok{\textquotesingle{}Outra\textquotesingle{}}\NormalTok{))}\SpecialCharTok{\%\textgreater{}\%}
\NormalTok{  ggplot2}\SpecialCharTok{::}\FunctionTok{ggplot}\NormalTok{(}
\NormalTok{    ggplot2}\SpecialCharTok{::}\FunctionTok{aes}\NormalTok{(}\AttributeTok{x =} \StringTok{\textasciigrave{}}\AttributeTok{(\textquotesingle{}P1\_b \textquotesingle{}, \textquotesingle{}Genero\textquotesingle{})}\StringTok{\textasciigrave{}}\NormalTok{, }\AttributeTok{y =}\NormalTok{ Prop,}
                 \AttributeTok{fill =} \StringTok{\textasciigrave{}}\AttributeTok{(\textquotesingle{}P1\_a\_a \textquotesingle{}, \textquotesingle{}Faixa idade\textquotesingle{})}\StringTok{\textasciigrave{}}\NormalTok{)) }\SpecialCharTok{+}
\NormalTok{  ggplot2}\SpecialCharTok{::}\FunctionTok{geom\_col}\NormalTok{( }\AttributeTok{color=}\StringTok{"white"}\NormalTok{,}
                     \AttributeTok{position =}\NormalTok{ ggplot2}\SpecialCharTok{::}\FunctionTok{position\_fill}\NormalTok{()) }\SpecialCharTok{+}
\NormalTok{  ggrepel}\SpecialCharTok{::}\FunctionTok{geom\_label\_repel}\NormalTok{(ggplot2}\SpecialCharTok{::}\FunctionTok{aes}\NormalTok{(}
    \AttributeTok{label =}\NormalTok{ scales}\SpecialCharTok{::}\FunctionTok{percent}\NormalTok{(Prop)),}
    \AttributeTok{fontface =} \StringTok{\textquotesingle{}bold\textquotesingle{}}\NormalTok{,}
    \AttributeTok{hjust=}\DecValTok{2}\NormalTok{,}
    \AttributeTok{position =}\NormalTok{ ggplot2}\SpecialCharTok{::}\FunctionTok{position\_stack}\NormalTok{(}\AttributeTok{vjust =}\NormalTok{ .}\DecValTok{5}\NormalTok{),}
    \AttributeTok{size=}\FloatTok{3.5}\NormalTok{) }\SpecialCharTok{+}
\NormalTok{  ggplot2}\SpecialCharTok{::}\FunctionTok{ggtitle}\NormalTok{(}\StringTok{\textquotesingle{}Figura 13: Faixa de Idade x Nível de Cargo x Gênero\textquotesingle{}}\NormalTok{)}\SpecialCharTok{+}
\NormalTok{  ggplot2}\SpecialCharTok{::}\FunctionTok{theme\_void}\NormalTok{()}\SpecialCharTok{+}
\NormalTok{  ggplot2}\SpecialCharTok{::}\FunctionTok{scale\_fill\_brewer}\NormalTok{(}\AttributeTok{type =} \StringTok{"seq"}\NormalTok{, }\AttributeTok{palette =} \StringTok{"Reds"}\NormalTok{)}\SpecialCharTok{+}
\NormalTok{  ggplot2}\SpecialCharTok{::}\FunctionTok{xlab}\NormalTok{(}\StringTok{\textquotesingle{}Gênero\textquotesingle{}}\NormalTok{)}\SpecialCharTok{+}
\NormalTok{  ggplot2}\SpecialCharTok{::}\FunctionTok{theme}\NormalTok{(}\AttributeTok{legend.position=}\StringTok{"bottom"}\NormalTok{,}
                 \AttributeTok{plot.title=}\NormalTok{ggplot2}\SpecialCharTok{::}\FunctionTok{element\_text}\NormalTok{(}\AttributeTok{face=}\StringTok{\textquotesingle{}bold.italic\textquotesingle{}}\NormalTok{,}
                                                  \AttributeTok{hjust =} \FloatTok{0.5}\NormalTok{, }\AttributeTok{size=}\DecValTok{20}\NormalTok{),}
                 \AttributeTok{axis.text.y=}\NormalTok{ggplot2}\SpecialCharTok{::}\FunctionTok{element\_blank}\NormalTok{(),}
                 \AttributeTok{axis.title.y=}\NormalTok{ggplot2}\SpecialCharTok{::}\FunctionTok{element\_blank}\NormalTok{(),}
                 \AttributeTok{axis.title.x=}\NormalTok{ggplot2}\SpecialCharTok{::}\FunctionTok{element\_blank}\NormalTok{(),}
                 \AttributeTok{axis.text.x =}\NormalTok{ggplot2}\SpecialCharTok{::}\FunctionTok{element\_text}\NormalTok{(}\AttributeTok{face=}\StringTok{\textquotesingle{}bold\textquotesingle{}}\NormalTok{, }\AttributeTok{size=}\DecValTok{12}\NormalTok{),}
                 \AttributeTok{legend.title=}\NormalTok{ggplot2}\SpecialCharTok{::}\FunctionTok{element\_blank}\NormalTok{())}\SpecialCharTok{+}
\NormalTok{  ggplot2}\SpecialCharTok{::}\FunctionTok{facet\_grid}\NormalTok{(}\SpecialCharTok{\textasciitilde{}}\StringTok{\textasciigrave{}}\AttributeTok{(\textquotesingle{}P2\_g \textquotesingle{}, \textquotesingle{}Nivel\textquotesingle{})}\StringTok{\textasciigrave{}}\NormalTok{)}
\end{Highlighting}
\end{Shaded}

\includegraphics{relatorio_files/figure-latex/unnamed-chunk-9-1.pdf}

\begin{Shaded}
\begin{Highlighting}[]
\CommentTok{\#Faixas de idade por gênero e atuação}
\NormalTok{df}\SpecialCharTok{\%\textgreater{}\%}
\NormalTok{  dplyr}\SpecialCharTok{::}\FunctionTok{count}\NormalTok{(}\StringTok{\textasciigrave{}}\AttributeTok{(\textquotesingle{}P4\_a \textquotesingle{}, \textquotesingle{}Atuacao\textquotesingle{})}\StringTok{\textasciigrave{}}\NormalTok{,}\StringTok{\textasciigrave{}}\AttributeTok{(\textquotesingle{}P1\_b \textquotesingle{}, \textquotesingle{}Genero\textquotesingle{})}\StringTok{\textasciigrave{}}\NormalTok{,}
               \StringTok{\textasciigrave{}}\AttributeTok{(\textquotesingle{}P1\_a\_a \textquotesingle{}, \textquotesingle{}Faixa idade\textquotesingle{})}\StringTok{\textasciigrave{}}\NormalTok{) }\SpecialCharTok{\%\textgreater{}\%}
\NormalTok{  dplyr}\SpecialCharTok{::}\FunctionTok{group\_by}\NormalTok{(}\StringTok{\textasciigrave{}}\AttributeTok{(\textquotesingle{}P4\_a \textquotesingle{}, \textquotesingle{}Atuacao\textquotesingle{})}\StringTok{\textasciigrave{}}\NormalTok{,}\StringTok{\textasciigrave{}}\AttributeTok{(\textquotesingle{}P1\_b \textquotesingle{}, \textquotesingle{}Genero\textquotesingle{})}\StringTok{\textasciigrave{}}\NormalTok{) }\SpecialCharTok{\%\textgreater{}\%}
\NormalTok{  dplyr}\SpecialCharTok{::}\FunctionTok{mutate}\NormalTok{(}\AttributeTok{Prop =}\NormalTok{ n}\SpecialCharTok{/}\FunctionTok{sum}\NormalTok{(n))}\SpecialCharTok{\%\textgreater{}\%}
\NormalTok{  dplyr}\SpecialCharTok{::}\FunctionTok{filter}\NormalTok{(}\StringTok{\textasciigrave{}}\AttributeTok{(\textquotesingle{}P4\_a \textquotesingle{}, \textquotesingle{}Atuacao\textquotesingle{})}\StringTok{\textasciigrave{}}\SpecialCharTok{\%notin\%}\FunctionTok{c}\NormalTok{(}\StringTok{\textquotesingle{}Outra\textquotesingle{}}\NormalTok{,}\StringTok{\textquotesingle{}Gestor\textquotesingle{}}\NormalTok{))}\SpecialCharTok{\%\textgreater{}\%}
\NormalTok{  ggplot2}\SpecialCharTok{::}\FunctionTok{ggplot}\NormalTok{(}
\NormalTok{    ggplot2}\SpecialCharTok{::}\FunctionTok{aes}\NormalTok{(}\AttributeTok{x =} \StringTok{\textasciigrave{}}\AttributeTok{(\textquotesingle{}P1\_b \textquotesingle{}, \textquotesingle{}Genero\textquotesingle{})}\StringTok{\textasciigrave{}}\NormalTok{, }\AttributeTok{y =}\NormalTok{ Prop,}
                 \AttributeTok{fill =} \StringTok{\textasciigrave{}}\AttributeTok{(\textquotesingle{}P1\_a\_a \textquotesingle{}, \textquotesingle{}Faixa idade\textquotesingle{})}\StringTok{\textasciigrave{}}\NormalTok{)) }\SpecialCharTok{+}
\NormalTok{  ggplot2}\SpecialCharTok{::}\FunctionTok{geom\_col}\NormalTok{( }\AttributeTok{color=}\StringTok{"white"}\NormalTok{,}
                     \AttributeTok{position =}\NormalTok{ ggplot2}\SpecialCharTok{::}\FunctionTok{position\_fill}\NormalTok{()) }\SpecialCharTok{+}
\NormalTok{  ggrepel}\SpecialCharTok{::}\FunctionTok{geom\_label\_repel}\NormalTok{(ggplot2}\SpecialCharTok{::}\FunctionTok{aes}\NormalTok{(}
    \AttributeTok{label =}\NormalTok{ scales}\SpecialCharTok{::}\FunctionTok{percent}\NormalTok{(Prop)),}
    \AttributeTok{fontface =} \StringTok{\textquotesingle{}bold\textquotesingle{}}\NormalTok{,}
    \AttributeTok{hjust=}\DecValTok{2}\NormalTok{,}
    \AttributeTok{position =}\NormalTok{ ggplot2}\SpecialCharTok{::}\FunctionTok{position\_stack}\NormalTok{(}\AttributeTok{vjust =}\NormalTok{ .}\DecValTok{5}\NormalTok{),}
    \AttributeTok{size=}\FloatTok{3.5}\NormalTok{) }\SpecialCharTok{+}
\NormalTok{  ggplot2}\SpecialCharTok{::}\FunctionTok{ggtitle}\NormalTok{(}\StringTok{\textquotesingle{}Figura 14: Faixa de Idade x Atuação x Gênero\textquotesingle{}}\NormalTok{)}\SpecialCharTok{+}
\NormalTok{  ggplot2}\SpecialCharTok{::}\FunctionTok{theme\_void}\NormalTok{()}\SpecialCharTok{+}
\NormalTok{  ggplot2}\SpecialCharTok{::}\FunctionTok{scale\_fill\_brewer}\NormalTok{(}\AttributeTok{type =} \StringTok{"seq"}\NormalTok{, }\AttributeTok{palette =} \StringTok{"Reds"}\NormalTok{)}\SpecialCharTok{+}
\NormalTok{  ggplot2}\SpecialCharTok{::}\FunctionTok{xlab}\NormalTok{(}\StringTok{\textquotesingle{}Gênero\textquotesingle{}}\NormalTok{)}\SpecialCharTok{+}
\NormalTok{  ggplot2}\SpecialCharTok{::}\FunctionTok{theme}\NormalTok{(}\AttributeTok{legend.position=}\StringTok{"bottom"}\NormalTok{,}
                 \AttributeTok{plot.title=}\NormalTok{ggplot2}\SpecialCharTok{::}\FunctionTok{element\_text}\NormalTok{(}\AttributeTok{face=}\StringTok{\textquotesingle{}bold.italic\textquotesingle{}}\NormalTok{,}
                                                  \AttributeTok{hjust =} \FloatTok{0.5}\NormalTok{, }\AttributeTok{size=}\DecValTok{20}\NormalTok{),}
                 \AttributeTok{axis.text.y=}\NormalTok{ggplot2}\SpecialCharTok{::}\FunctionTok{element\_blank}\NormalTok{(),}
                 \AttributeTok{axis.title.y=}\NormalTok{ggplot2}\SpecialCharTok{::}\FunctionTok{element\_blank}\NormalTok{(),}
                 \AttributeTok{axis.title.x=}\NormalTok{ggplot2}\SpecialCharTok{::}\FunctionTok{element\_blank}\NormalTok{(),}
                 \AttributeTok{axis.text.x =}\NormalTok{ggplot2}\SpecialCharTok{::}\FunctionTok{element\_text}\NormalTok{(}\AttributeTok{face=}\StringTok{\textquotesingle{}bold\textquotesingle{}}\NormalTok{, }\AttributeTok{size=}\DecValTok{12}\NormalTok{),}
                 \AttributeTok{legend.title=}\NormalTok{ggplot2}\SpecialCharTok{::}\FunctionTok{element\_blank}\NormalTok{())}\SpecialCharTok{+}
\NormalTok{  ggplot2}\SpecialCharTok{::}\FunctionTok{facet\_grid}\NormalTok{(}\SpecialCharTok{\textasciitilde{}}\StringTok{\textasciigrave{}}\AttributeTok{(\textquotesingle{}P4\_a \textquotesingle{}, \textquotesingle{}Atuacao\textquotesingle{})}\StringTok{\textasciigrave{}}\NormalTok{)}
\end{Highlighting}
\end{Shaded}

\includegraphics{relatorio_files/figure-latex/unnamed-chunk-9-2.pdf}

Como dito anteriormente, as mulheres da área de dados estão ganhando
menos nos cargos sênior e de gestão. Em partes, isso pode ter relação
com a maior concentração de mulheres na função que, segundo o relatório
da State of Data Brazil 2021 - Figura 28, é a que menos paga (análise de
dados) em todos os níveis de cargo, principalmente no nível sênior, com
mais de 61\% das profissionais deste nível estando nessa função contra
menos de 42\% dos homens sênior, além da menor concentração delas no
nível sênior da engenharia de dados (que, segundo a mesma análise do
relatório, é a que melhor paga no nível sênior). Além disso, levando em
conta as considerações feitas no início desta apresentação (apenas
pessoas, dos gêneros masculino ou feminino, cuja situação de trabalho
era ``Empreendedor ou Empregado (CNPJ)'' ou ``Empregado (CLT)''), não há
mudança significativa entre as diferenças de remuneração por gênero e
nível de cargo.

\begin{Shaded}
\begin{Highlighting}[]
\CommentTok{\#Atuação por gênero e Nível de cargo}
\NormalTok{df}\SpecialCharTok{\%\textgreater{}\%}
\NormalTok{  dplyr}\SpecialCharTok{::}\FunctionTok{count}\NormalTok{(}\StringTok{\textasciigrave{}}\AttributeTok{(\textquotesingle{}P4\_a \textquotesingle{}, \textquotesingle{}Atuacao\textquotesingle{})}\StringTok{\textasciigrave{}}\NormalTok{,}\StringTok{\textasciigrave{}}\AttributeTok{(\textquotesingle{}P1\_b \textquotesingle{}, \textquotesingle{}Genero\textquotesingle{})}\StringTok{\textasciigrave{}}\NormalTok{, }\StringTok{\textasciigrave{}}\AttributeTok{(\textquotesingle{}P2\_g \textquotesingle{}, \textquotesingle{}Nivel\textquotesingle{})}\StringTok{\textasciigrave{}}\NormalTok{) }\SpecialCharTok{\%\textgreater{}\%}
\NormalTok{  dplyr}\SpecialCharTok{::}\FunctionTok{group\_by}\NormalTok{(}\StringTok{\textasciigrave{}}\AttributeTok{(\textquotesingle{}P1\_b \textquotesingle{}, \textquotesingle{}Genero\textquotesingle{})}\StringTok{\textasciigrave{}}\NormalTok{,}\StringTok{\textasciigrave{}}\AttributeTok{(\textquotesingle{}P2\_g \textquotesingle{}, \textquotesingle{}Nivel\textquotesingle{})}\StringTok{\textasciigrave{}}\NormalTok{) }\SpecialCharTok{\%\textgreater{}\%}
\NormalTok{  dplyr}\SpecialCharTok{::}\FunctionTok{mutate}\NormalTok{(}\AttributeTok{Prop =}\NormalTok{ n}\SpecialCharTok{/}\FunctionTok{sum}\NormalTok{(n))}\SpecialCharTok{\%\textgreater{}\%}
\NormalTok{  dplyr}\SpecialCharTok{::}\FunctionTok{filter}\NormalTok{(}\StringTok{\textasciigrave{}}\AttributeTok{(\textquotesingle{}P2\_g \textquotesingle{}, \textquotesingle{}Nivel\textquotesingle{})}\StringTok{\textasciigrave{}}\SpecialCharTok{!=}\StringTok{\textquotesingle{}Gestor\textquotesingle{}}\NormalTok{)}\SpecialCharTok{\%\textgreater{}\%}
\NormalTok{  ggplot2}\SpecialCharTok{::}\FunctionTok{ggplot}\NormalTok{(}
\NormalTok{    ggplot2}\SpecialCharTok{::}\FunctionTok{aes}\NormalTok{(}\AttributeTok{x =} \StringTok{\textasciigrave{}}\AttributeTok{(\textquotesingle{}P1\_b \textquotesingle{}, \textquotesingle{}Genero\textquotesingle{})}\StringTok{\textasciigrave{}}\NormalTok{, }\AttributeTok{y =}\NormalTok{ Prop,}
                 \AttributeTok{fill =} \StringTok{\textasciigrave{}}\AttributeTok{(\textquotesingle{}P4\_a \textquotesingle{}, \textquotesingle{}Atuacao\textquotesingle{})}\StringTok{\textasciigrave{}}\NormalTok{)) }\SpecialCharTok{+}
\NormalTok{  ggplot2}\SpecialCharTok{::}\FunctionTok{geom\_col}\NormalTok{( }\AttributeTok{color=}\StringTok{"white"}\NormalTok{,}
                     \AttributeTok{position =}\NormalTok{ ggplot2}\SpecialCharTok{::}\FunctionTok{position\_fill}\NormalTok{()) }\SpecialCharTok{+}
\NormalTok{  ggrepel}\SpecialCharTok{::}\FunctionTok{geom\_label\_repel}\NormalTok{(ggplot2}\SpecialCharTok{::}\FunctionTok{aes}\NormalTok{(}
    \AttributeTok{label =}\NormalTok{ scales}\SpecialCharTok{::}\FunctionTok{percent}\NormalTok{(Prop)),}
    \AttributeTok{fontface =} \StringTok{\textquotesingle{}bold\textquotesingle{}}\NormalTok{,}
    \AttributeTok{hjust=}\DecValTok{2}\NormalTok{,}
    \AttributeTok{position =}\NormalTok{ ggplot2}\SpecialCharTok{::}\FunctionTok{position\_stack}\NormalTok{(}\AttributeTok{vjust =}\NormalTok{ .}\DecValTok{5}\NormalTok{),}
    \AttributeTok{size=}\FloatTok{3.5}\NormalTok{) }\SpecialCharTok{+}
\NormalTok{  ggplot2}\SpecialCharTok{::}\FunctionTok{ggtitle}\NormalTok{(}\StringTok{\textquotesingle{}Figura 15: Atuação x Nível de Cargo x Gênero\textquotesingle{}}\NormalTok{)}\SpecialCharTok{+}
\NormalTok{  ggplot2}\SpecialCharTok{::}\FunctionTok{theme\_void}\NormalTok{()}\SpecialCharTok{+}
\NormalTok{  ggplot2}\SpecialCharTok{::}\FunctionTok{scale\_fill\_brewer}\NormalTok{(}\AttributeTok{type =} \StringTok{"seq"}\NormalTok{, }\AttributeTok{palette =} \StringTok{"Spectral"}\NormalTok{)}\SpecialCharTok{+}
\NormalTok{  ggplot2}\SpecialCharTok{::}\FunctionTok{xlab}\NormalTok{(}\StringTok{\textquotesingle{}Gênero\textquotesingle{}}\NormalTok{)}\SpecialCharTok{+}
\NormalTok{  ggplot2}\SpecialCharTok{::}\FunctionTok{theme}\NormalTok{(}\AttributeTok{legend.position=}\StringTok{"bottom"}\NormalTok{,}
                 \AttributeTok{plot.title=}\NormalTok{ggplot2}\SpecialCharTok{::}\FunctionTok{element\_text}\NormalTok{(}\AttributeTok{face=}\StringTok{\textquotesingle{}bold.italic\textquotesingle{}}\NormalTok{,}
                                                  \AttributeTok{hjust =} \FloatTok{0.5}\NormalTok{, }\AttributeTok{size=}\DecValTok{20}\NormalTok{),}
                 \AttributeTok{axis.text.y=}\NormalTok{ggplot2}\SpecialCharTok{::}\FunctionTok{element\_blank}\NormalTok{(),}
                 \AttributeTok{axis.title.y=}\NormalTok{ggplot2}\SpecialCharTok{::}\FunctionTok{element\_blank}\NormalTok{(),}
                 \AttributeTok{axis.title.x=}\NormalTok{ggplot2}\SpecialCharTok{::}\FunctionTok{element\_blank}\NormalTok{(),}
                 \AttributeTok{axis.text.x =}\NormalTok{ggplot2}\SpecialCharTok{::}\FunctionTok{element\_text}\NormalTok{(}\AttributeTok{face=}\StringTok{\textquotesingle{}bold\textquotesingle{}}\NormalTok{, }\AttributeTok{size=}\DecValTok{12}\NormalTok{),}
                 \AttributeTok{legend.title=}\NormalTok{ggplot2}\SpecialCharTok{::}\FunctionTok{element\_blank}\NormalTok{())}\SpecialCharTok{+}
\NormalTok{  ggplot2}\SpecialCharTok{::}\FunctionTok{facet\_grid}\NormalTok{(}\SpecialCharTok{\textasciitilde{}}\StringTok{\textasciigrave{}}\AttributeTok{(\textquotesingle{}P2\_g \textquotesingle{}, \textquotesingle{}Nivel\textquotesingle{})}\StringTok{\textasciigrave{}}\NormalTok{)}
\end{Highlighting}
\end{Shaded}

\includegraphics{relatorio_files/figure-latex/unnamed-chunk-10-1.pdf}

\begin{Shaded}
\begin{Highlighting}[]
\CommentTok{\#Faixa salarial por gênero e nível de cargo}
\NormalTok{df}\SpecialCharTok{\%\textgreater{}\%}
\NormalTok{  dplyr}\SpecialCharTok{::}\FunctionTok{count}\NormalTok{(}\StringTok{\textasciigrave{}}\AttributeTok{(\textquotesingle{}P2\_g \textquotesingle{}, \textquotesingle{}Nivel\textquotesingle{})}\StringTok{\textasciigrave{}}\NormalTok{,}\StringTok{\textasciigrave{}}\AttributeTok{(\textquotesingle{}P1\_b \textquotesingle{}, \textquotesingle{}Genero\textquotesingle{})}\StringTok{\textasciigrave{}}\NormalTok{, }\StringTok{\textasciigrave{}}\AttributeTok{(\textquotesingle{}P2\_h \textquotesingle{}, \textquotesingle{}Faixa salarial\textquotesingle{})}\StringTok{\textasciigrave{}}\NormalTok{) }\SpecialCharTok{\%\textgreater{}\%}
\NormalTok{  dplyr}\SpecialCharTok{::}\FunctionTok{group\_by}\NormalTok{(}\StringTok{\textasciigrave{}}\AttributeTok{(\textquotesingle{}P2\_g \textquotesingle{}, \textquotesingle{}Nivel\textquotesingle{})}\StringTok{\textasciigrave{}}\NormalTok{,}\StringTok{\textasciigrave{}}\AttributeTok{(\textquotesingle{}P1\_b \textquotesingle{}, \textquotesingle{}Genero\textquotesingle{})}\StringTok{\textasciigrave{}}\NormalTok{) }\SpecialCharTok{\%\textgreater{}\%}
\NormalTok{  dplyr}\SpecialCharTok{::}\FunctionTok{mutate}\NormalTok{(}\AttributeTok{Prop =}\NormalTok{ n}\SpecialCharTok{/}\FunctionTok{sum}\NormalTok{(n))}\SpecialCharTok{\%\textgreater{}\%}
\NormalTok{  ggplot2}\SpecialCharTok{::}\FunctionTok{ggplot}\NormalTok{(}
\NormalTok{    ggplot2}\SpecialCharTok{::}\FunctionTok{aes}\NormalTok{(}\AttributeTok{x =} \StringTok{\textasciigrave{}}\AttributeTok{(\textquotesingle{}P1\_b \textquotesingle{}, \textquotesingle{}Genero\textquotesingle{})}\StringTok{\textasciigrave{}}\NormalTok{, }\AttributeTok{y =}\NormalTok{ Prop,}
                 \AttributeTok{fill =} \StringTok{\textasciigrave{}}\AttributeTok{(\textquotesingle{}P2\_h \textquotesingle{}, \textquotesingle{}Faixa salarial\textquotesingle{})}\StringTok{\textasciigrave{}}\NormalTok{)) }\SpecialCharTok{+}
\NormalTok{  ggplot2}\SpecialCharTok{::}\FunctionTok{geom\_col}\NormalTok{( }\AttributeTok{color=}\StringTok{"white"}\NormalTok{,}
                     \AttributeTok{position =}\NormalTok{ ggplot2}\SpecialCharTok{::}\FunctionTok{position\_fill}\NormalTok{()) }\SpecialCharTok{+}
\NormalTok{  ggrepel}\SpecialCharTok{::}\FunctionTok{geom\_label\_repel}\NormalTok{(ggplot2}\SpecialCharTok{::}\FunctionTok{aes}\NormalTok{(}
    \AttributeTok{label =}\NormalTok{ scales}\SpecialCharTok{::}\FunctionTok{percent}\NormalTok{(Prop)),}
    \AttributeTok{fontface =} \StringTok{\textquotesingle{}bold\textquotesingle{}}\NormalTok{,}
    \AttributeTok{hjust=}\DecValTok{2}\NormalTok{,}
    \AttributeTok{position =}\NormalTok{ ggplot2}\SpecialCharTok{::}\FunctionTok{position\_stack}\NormalTok{(}\AttributeTok{vjust =}\NormalTok{ .}\DecValTok{5}\NormalTok{),}
    \AttributeTok{size=}\FloatTok{3.5}\NormalTok{) }\SpecialCharTok{+}
\NormalTok{  ggplot2}\SpecialCharTok{::}\FunctionTok{labs}\NormalTok{(}\AttributeTok{title =} \StringTok{\textquotesingle{}Figura 16: Faixa Salarial x Nível de Cargo x Gênero\textquotesingle{}}\NormalTok{)}\SpecialCharTok{+}
\NormalTok{  ggplot2}\SpecialCharTok{::}\FunctionTok{theme\_void}\NormalTok{()}\SpecialCharTok{+}
\NormalTok{  ggplot2}\SpecialCharTok{::}\FunctionTok{scale\_fill\_brewer}\NormalTok{(}\AttributeTok{type =} \StringTok{"seq"}\NormalTok{, }\AttributeTok{palette =} \StringTok{"YlOrRd"}\NormalTok{)}\SpecialCharTok{+}
\NormalTok{  ggplot2}\SpecialCharTok{::}\FunctionTok{xlab}\NormalTok{(}\StringTok{\textquotesingle{}Gênero\textquotesingle{}}\NormalTok{)}\SpecialCharTok{+}
\NormalTok{  ggplot2}\SpecialCharTok{::}\FunctionTok{theme}\NormalTok{(}\AttributeTok{legend.position=}\StringTok{"bottom"}\NormalTok{,}
                 \AttributeTok{plot.title=}\NormalTok{ggplot2}\SpecialCharTok{::}\FunctionTok{element\_text}\NormalTok{(}\AttributeTok{face=}\StringTok{\textquotesingle{}bold.italic\textquotesingle{}}\NormalTok{,}
                                                  \AttributeTok{hjust =} \FloatTok{0.5}\NormalTok{, }\AttributeTok{size=}\DecValTok{20}\NormalTok{),}
                 \AttributeTok{axis.text.y=}\NormalTok{ggplot2}\SpecialCharTok{::}\FunctionTok{element\_blank}\NormalTok{(),}
                 \AttributeTok{axis.title.y=}\NormalTok{ggplot2}\SpecialCharTok{::}\FunctionTok{element\_blank}\NormalTok{(),}
                 \AttributeTok{axis.title.x=}\NormalTok{ggplot2}\SpecialCharTok{::}\FunctionTok{element\_blank}\NormalTok{(),}
                 \AttributeTok{axis.text.x =}\NormalTok{ggplot2}\SpecialCharTok{::}\FunctionTok{element\_text}\NormalTok{(}\AttributeTok{face=}\StringTok{\textquotesingle{}bold\textquotesingle{}}\NormalTok{, }\AttributeTok{size=}\DecValTok{12}\NormalTok{),}
                 \AttributeTok{legend.title=}\NormalTok{ggplot2}\SpecialCharTok{::}\FunctionTok{element\_blank}\NormalTok{())}\SpecialCharTok{+}
\NormalTok{  ggplot2}\SpecialCharTok{::}\FunctionTok{facet\_grid}\NormalTok{(}\SpecialCharTok{\textasciitilde{}}\StringTok{\textasciigrave{}}\AttributeTok{(\textquotesingle{}P2\_g \textquotesingle{}, \textquotesingle{}Nivel\textquotesingle{})}\StringTok{\textasciigrave{}}\NormalTok{)}
\end{Highlighting}
\end{Shaded}

\begin{verbatim}
## Warning in RColorBrewer::brewer.pal(n, pal): n too large, allowed maximum for palette YlOrRd is 9
## Returning the palette you asked for with that many colors
\end{verbatim}

\begin{verbatim}
## Warning: ggrepel: 5 unlabeled data points (too many overlaps). Consider
## increasing max.overlaps
\end{verbatim}

\includegraphics{relatorio_files/figure-latex/unnamed-chunk-10-2.pdf}

No entanto, ao verificar as faixas salariais em cada atuação,
percebemos, entre os cientistas de dados, uma considerável desvantagem
das mulheres em relação aos homens nas faixas salariais acima de R\$ 16
mil (1,6\% x 9,6\%), rendimentos geralmente ganhos pelos profissionais
sênior, o que indica que há outros fatores que causam isso, além da
maior proporção de analistas de dados do gênero feminino. Esta diferença
é ainda maior entre os engenheiros de dados, causada pela excassez de
engenheiras de dados sênior, mostrada na Figura 15. Apenas entre os
analistas de dados é que parece haver equilíbrio em termos de salário
entre os 2 gêneros. Com a exclusão dos engenheiros de dados sênior, o
balanceamento salarial entre os gêneros, nesta função, aumenta
consideravelmente, mas, ainda sim, quase 6\% dos homens entrevistados se
encontram nas faixas salariais acima dos 12 mil reais, enquanto que
nenhuma mulher (0\%) se encontra em tal situação.

\begin{Shaded}
\begin{Highlighting}[]
\CommentTok{\#Faixa salarial por gênero e atuação}
\NormalTok{df}\SpecialCharTok{\%\textgreater{}\%}
\NormalTok{  dplyr}\SpecialCharTok{::}\FunctionTok{count}\NormalTok{(}\StringTok{\textasciigrave{}}\AttributeTok{(\textquotesingle{}P4\_a \textquotesingle{}, \textquotesingle{}Atuacao\textquotesingle{})}\StringTok{\textasciigrave{}}\NormalTok{,}\StringTok{\textasciigrave{}}\AttributeTok{(\textquotesingle{}P1\_b \textquotesingle{}, \textquotesingle{}Genero\textquotesingle{})}\StringTok{\textasciigrave{}}\NormalTok{, }\StringTok{\textasciigrave{}}\AttributeTok{(\textquotesingle{}P2\_h \textquotesingle{}, \textquotesingle{}Faixa salarial\textquotesingle{})}\StringTok{\textasciigrave{}}\NormalTok{) }\SpecialCharTok{\%\textgreater{}\%}
\NormalTok{  dplyr}\SpecialCharTok{::}\FunctionTok{group\_by}\NormalTok{(}\StringTok{\textasciigrave{}}\AttributeTok{(\textquotesingle{}P4\_a \textquotesingle{}, \textquotesingle{}Atuacao\textquotesingle{})}\StringTok{\textasciigrave{}}\NormalTok{,}\StringTok{\textasciigrave{}}\AttributeTok{(\textquotesingle{}P1\_b \textquotesingle{}, \textquotesingle{}Genero\textquotesingle{})}\StringTok{\textasciigrave{}}\NormalTok{) }\SpecialCharTok{\%\textgreater{}\%}
\NormalTok{  dplyr}\SpecialCharTok{::}\FunctionTok{mutate}\NormalTok{(}\AttributeTok{Prop =}\NormalTok{ n}\SpecialCharTok{/}\FunctionTok{sum}\NormalTok{(n))}\SpecialCharTok{\%\textgreater{}\%}
\NormalTok{  dplyr}\SpecialCharTok{::}\FunctionTok{filter}\NormalTok{(}\StringTok{\textasciigrave{}}\AttributeTok{(\textquotesingle{}P4\_a \textquotesingle{}, \textquotesingle{}Atuacao\textquotesingle{})}\StringTok{\textasciigrave{}}\SpecialCharTok{\%notin\%}\FunctionTok{c}\NormalTok{(}\StringTok{\textquotesingle{}Outra\textquotesingle{}}\NormalTok{, }\StringTok{\textquotesingle{}Gestor\textquotesingle{}}\NormalTok{))}\SpecialCharTok{\%\textgreater{}\%}
\NormalTok{  ggplot2}\SpecialCharTok{::}\FunctionTok{ggplot}\NormalTok{(}
\NormalTok{    ggplot2}\SpecialCharTok{::}\FunctionTok{aes}\NormalTok{(}\AttributeTok{x =} \StringTok{\textasciigrave{}}\AttributeTok{(\textquotesingle{}P1\_b \textquotesingle{}, \textquotesingle{}Genero\textquotesingle{})}\StringTok{\textasciigrave{}}\NormalTok{, }\AttributeTok{y =}\NormalTok{ Prop,}
                 \AttributeTok{fill =} \StringTok{\textasciigrave{}}\AttributeTok{(\textquotesingle{}P2\_h \textquotesingle{}, \textquotesingle{}Faixa salarial\textquotesingle{})}\StringTok{\textasciigrave{}}\NormalTok{)) }\SpecialCharTok{+}
\NormalTok{  ggplot2}\SpecialCharTok{::}\FunctionTok{geom\_col}\NormalTok{( }\AttributeTok{color=}\StringTok{"white"}\NormalTok{,}
                     \AttributeTok{position =}\NormalTok{ ggplot2}\SpecialCharTok{::}\FunctionTok{position\_fill}\NormalTok{()) }\SpecialCharTok{+}
\NormalTok{  ggrepel}\SpecialCharTok{::}\FunctionTok{geom\_label\_repel}\NormalTok{(ggplot2}\SpecialCharTok{::}\FunctionTok{aes}\NormalTok{(}
    \AttributeTok{label =}\NormalTok{ scales}\SpecialCharTok{::}\FunctionTok{percent}\NormalTok{(Prop)),}
    \AttributeTok{fontface =} \StringTok{\textquotesingle{}bold\textquotesingle{}}\NormalTok{,}
    \AttributeTok{hjust=}\DecValTok{2}\NormalTok{,}
    \AttributeTok{position =}\NormalTok{ ggplot2}\SpecialCharTok{::}\FunctionTok{position\_stack}\NormalTok{(}\AttributeTok{vjust =}\NormalTok{ .}\DecValTok{5}\NormalTok{),}
    \AttributeTok{size=}\FloatTok{3.5}\NormalTok{) }\SpecialCharTok{+}
\NormalTok{  ggplot2}\SpecialCharTok{::}\FunctionTok{labs}\NormalTok{(}\AttributeTok{title =} \StringTok{\textquotesingle{}Figura 17: Faixa Salarial x Atuação x Gênero\textquotesingle{}}\NormalTok{)}\SpecialCharTok{+}
\NormalTok{  ggplot2}\SpecialCharTok{::}\FunctionTok{theme\_void}\NormalTok{()}\SpecialCharTok{+}
\NormalTok{  ggplot2}\SpecialCharTok{::}\FunctionTok{scale\_fill\_brewer}\NormalTok{(}\AttributeTok{type =} \StringTok{"seq"}\NormalTok{, }\AttributeTok{palette =} \StringTok{"YlOrRd"}\NormalTok{)}\SpecialCharTok{+}
\NormalTok{  ggplot2}\SpecialCharTok{::}\FunctionTok{xlab}\NormalTok{(}\StringTok{\textquotesingle{}Gênero\textquotesingle{}}\NormalTok{)}\SpecialCharTok{+}
\NormalTok{  ggplot2}\SpecialCharTok{::}\FunctionTok{theme}\NormalTok{(}\AttributeTok{legend.position=}\StringTok{"bottom"}\NormalTok{,}
                 \AttributeTok{plot.title=}\NormalTok{ggplot2}\SpecialCharTok{::}\FunctionTok{element\_text}\NormalTok{(}\AttributeTok{face=}\StringTok{\textquotesingle{}bold.italic\textquotesingle{}}\NormalTok{,}
                                                  \AttributeTok{hjust =} \FloatTok{0.5}\NormalTok{, }\AttributeTok{size=}\DecValTok{20}\NormalTok{),}
                 \AttributeTok{axis.text.y=}\NormalTok{ggplot2}\SpecialCharTok{::}\FunctionTok{element\_blank}\NormalTok{(),}
                 \AttributeTok{axis.title.y=}\NormalTok{ggplot2}\SpecialCharTok{::}\FunctionTok{element\_blank}\NormalTok{(),}
                 \AttributeTok{axis.title.x=}\NormalTok{ggplot2}\SpecialCharTok{::}\FunctionTok{element\_blank}\NormalTok{(),}
                 \AttributeTok{axis.text.x =}\NormalTok{ggplot2}\SpecialCharTok{::}\FunctionTok{element\_text}\NormalTok{(}\AttributeTok{face=}\StringTok{\textquotesingle{}bold\textquotesingle{}}\NormalTok{, }\AttributeTok{size=}\DecValTok{12}\NormalTok{),}
                 \AttributeTok{legend.title=}\NormalTok{ggplot2}\SpecialCharTok{::}\FunctionTok{element\_blank}\NormalTok{())}\SpecialCharTok{+}
\NormalTok{  ggplot2}\SpecialCharTok{::}\FunctionTok{facet\_grid}\NormalTok{(}\SpecialCharTok{\textasciitilde{}}\StringTok{\textasciigrave{}}\AttributeTok{(\textquotesingle{}P4\_a \textquotesingle{}, \textquotesingle{}Atuacao\textquotesingle{})}\StringTok{\textasciigrave{}}\NormalTok{)}
\end{Highlighting}
\end{Shaded}

\begin{verbatim}
## Warning in RColorBrewer::brewer.pal(n, pal): n too large, allowed maximum for palette YlOrRd is 9
## Returning the palette you asked for with that many colors
\end{verbatim}

\includegraphics{relatorio_files/figure-latex/unnamed-chunk-11-1.pdf}

\begin{Shaded}
\begin{Highlighting}[]
\CommentTok{\#Faixa salarial por gênero, Engenheiros de Dados Júnior e Pleno}
\NormalTok{df}\SpecialCharTok{\%\textgreater{}\%}
\NormalTok{  dplyr}\SpecialCharTok{::}\FunctionTok{filter}\NormalTok{(}\StringTok{\textasciigrave{}}\AttributeTok{(\textquotesingle{}P4\_a \textquotesingle{}, \textquotesingle{}Atuacao\textquotesingle{})}\StringTok{\textasciigrave{}}\SpecialCharTok{\%in\%}\FunctionTok{c}\NormalTok{(}\StringTok{\textquotesingle{}Engenharia de Dados\textquotesingle{}}\NormalTok{) }\SpecialCharTok{\&}
                  \StringTok{\textasciigrave{}}\AttributeTok{(\textquotesingle{}P2\_g \textquotesingle{}, \textquotesingle{}Nivel\textquotesingle{})}\StringTok{\textasciigrave{}}\SpecialCharTok{!=}\StringTok{\textquotesingle{}Sênior\textquotesingle{}}\NormalTok{)}\SpecialCharTok{\%\textgreater{}\%}
\NormalTok{  dplyr}\SpecialCharTok{::}\FunctionTok{count}\NormalTok{(}\StringTok{\textasciigrave{}}\AttributeTok{(\textquotesingle{}P4\_a \textquotesingle{}, \textquotesingle{}Atuacao\textquotesingle{})}\StringTok{\textasciigrave{}}\NormalTok{,}\StringTok{\textasciigrave{}}\AttributeTok{(\textquotesingle{}P1\_b \textquotesingle{}, \textquotesingle{}Genero\textquotesingle{})}\StringTok{\textasciigrave{}}\NormalTok{, }\StringTok{\textasciigrave{}}\AttributeTok{(\textquotesingle{}P2\_h \textquotesingle{}, \textquotesingle{}Faixa salarial\textquotesingle{})}\StringTok{\textasciigrave{}}\NormalTok{) }\SpecialCharTok{\%\textgreater{}\%}
\NormalTok{  dplyr}\SpecialCharTok{::}\FunctionTok{group\_by}\NormalTok{(}\StringTok{\textasciigrave{}}\AttributeTok{(\textquotesingle{}P4\_a \textquotesingle{}, \textquotesingle{}Atuacao\textquotesingle{})}\StringTok{\textasciigrave{}}\NormalTok{,}\StringTok{\textasciigrave{}}\AttributeTok{(\textquotesingle{}P1\_b \textquotesingle{}, \textquotesingle{}Genero\textquotesingle{})}\StringTok{\textasciigrave{}}\NormalTok{) }\SpecialCharTok{\%\textgreater{}\%}
\NormalTok{  dplyr}\SpecialCharTok{::}\FunctionTok{mutate}\NormalTok{(}\AttributeTok{Prop =}\NormalTok{ n}\SpecialCharTok{/}\FunctionTok{sum}\NormalTok{(n))}\SpecialCharTok{\%\textgreater{}\%}
\NormalTok{  ggplot2}\SpecialCharTok{::}\FunctionTok{ggplot}\NormalTok{(}
\NormalTok{    ggplot2}\SpecialCharTok{::}\FunctionTok{aes}\NormalTok{(}\AttributeTok{x =} \StringTok{\textasciigrave{}}\AttributeTok{(\textquotesingle{}P1\_b \textquotesingle{}, \textquotesingle{}Genero\textquotesingle{})}\StringTok{\textasciigrave{}}\NormalTok{, }\AttributeTok{y =}\NormalTok{ Prop,}
                 \AttributeTok{fill =} \StringTok{\textasciigrave{}}\AttributeTok{(\textquotesingle{}P2\_h \textquotesingle{}, \textquotesingle{}Faixa salarial\textquotesingle{})}\StringTok{\textasciigrave{}}\NormalTok{)) }\SpecialCharTok{+}
\NormalTok{  ggplot2}\SpecialCharTok{::}\FunctionTok{geom\_col}\NormalTok{( }\AttributeTok{color=}\StringTok{"white"}\NormalTok{,}
                     \AttributeTok{position =}\NormalTok{ ggplot2}\SpecialCharTok{::}\FunctionTok{position\_fill}\NormalTok{()) }\SpecialCharTok{+}
\NormalTok{  ggrepel}\SpecialCharTok{::}\FunctionTok{geom\_label\_repel}\NormalTok{(ggplot2}\SpecialCharTok{::}\FunctionTok{aes}\NormalTok{(}
    \AttributeTok{label =}\NormalTok{ scales}\SpecialCharTok{::}\FunctionTok{percent}\NormalTok{(Prop)),}
    \AttributeTok{fontface =} \StringTok{\textquotesingle{}bold\textquotesingle{}}\NormalTok{,}
    \AttributeTok{hjust=}\DecValTok{2}\NormalTok{,}
    \AttributeTok{position =}\NormalTok{ ggplot2}\SpecialCharTok{::}\FunctionTok{position\_stack}\NormalTok{(}\AttributeTok{vjust =}\NormalTok{ .}\DecValTok{5}\NormalTok{),}
    \AttributeTok{size=}\FloatTok{3.5}\NormalTok{) }\SpecialCharTok{+}
\NormalTok{  ggplot2}\SpecialCharTok{::}\FunctionTok{labs}\NormalTok{(}\AttributeTok{title =} \StringTok{\textquotesingle{}Figura 18: Faixa Salarial dos Engenheiros de Dados Júnior e Pleno x Gênero\textquotesingle{}}\NormalTok{)}\SpecialCharTok{+}
\NormalTok{  ggplot2}\SpecialCharTok{::}\FunctionTok{theme\_void}\NormalTok{()}\SpecialCharTok{+}
\NormalTok{  ggplot2}\SpecialCharTok{::}\FunctionTok{scale\_fill\_brewer}\NormalTok{(}\AttributeTok{type =} \StringTok{"seq"}\NormalTok{, }\AttributeTok{palette =} \StringTok{"YlOrRd"}\NormalTok{)}\SpecialCharTok{+}
\NormalTok{  ggplot2}\SpecialCharTok{::}\FunctionTok{xlab}\NormalTok{(}\StringTok{\textquotesingle{}Gênero\textquotesingle{}}\NormalTok{)}\SpecialCharTok{+}
\NormalTok{  ggplot2}\SpecialCharTok{::}\FunctionTok{theme}\NormalTok{(}\AttributeTok{legend.position=}\StringTok{"bottom"}\NormalTok{,}
                 \AttributeTok{plot.title=}\NormalTok{ggplot2}\SpecialCharTok{::}\FunctionTok{element\_text}\NormalTok{(}\AttributeTok{face=}\StringTok{\textquotesingle{}bold.italic\textquotesingle{}}\NormalTok{,}
                                                  \AttributeTok{hjust =} \FloatTok{0.5}\NormalTok{, }\AttributeTok{size=}\DecValTok{20}\NormalTok{),}
                 \AttributeTok{axis.text.y=}\NormalTok{ggplot2}\SpecialCharTok{::}\FunctionTok{element\_blank}\NormalTok{(),}
                 \AttributeTok{axis.title.y=}\NormalTok{ggplot2}\SpecialCharTok{::}\FunctionTok{element\_blank}\NormalTok{(),}
                 \AttributeTok{axis.title.x=}\NormalTok{ggplot2}\SpecialCharTok{::}\FunctionTok{element\_blank}\NormalTok{(),}
                 \AttributeTok{axis.text.x =}\NormalTok{ggplot2}\SpecialCharTok{::}\FunctionTok{element\_text}\NormalTok{(}\AttributeTok{face=}\StringTok{\textquotesingle{}bold\textquotesingle{}}\NormalTok{, }\AttributeTok{size=}\DecValTok{12}\NormalTok{),}
                 \AttributeTok{legend.title=}\NormalTok{ggplot2}\SpecialCharTok{::}\FunctionTok{element\_blank}\NormalTok{())}\SpecialCharTok{+}
\NormalTok{  ggplot2}\SpecialCharTok{::}\FunctionTok{facet\_grid}\NormalTok{(}\SpecialCharTok{\textasciitilde{}}\StringTok{\textasciigrave{}}\AttributeTok{(\textquotesingle{}P4\_a \textquotesingle{}, \textquotesingle{}Atuacao\textquotesingle{})}\StringTok{\textasciigrave{}}\NormalTok{)}
\end{Highlighting}
\end{Shaded}

\includegraphics{relatorio_files/figure-latex/unnamed-chunk-11-2.pdf}

Em relação à situação de trabalho, 9,6\% das mulheres entrevistadas
estão no regime CNPJ contra 14,4\% dos homens (com o restante das
pessoas, de ambos os gêneros, se encontram no regime CLT). Além disso, a
distribuição dos CNPJ, tanto entre os níveis de cargo como entre as
funções, é parecida para homens e mulheres, sendo a diferença mais
significativa entre os cargos de gestão, onde quase 23\% das gestoras
são CNPJ enquanto que este mesmo percentual é de 29,8\% entre os
gestores homens. Portanto, este fator não parece explicar tanto as
diferenças salariais entre os 2 gêneros nos cargos de gestor e,
principalmente, de sênior.

\begin{Shaded}
\begin{Highlighting}[]
\NormalTok{df}\SpecialCharTok{\%\textgreater{}\%}
\NormalTok{  dplyr}\SpecialCharTok{::}\FunctionTok{count}\NormalTok{(}\StringTok{\textasciigrave{}}\AttributeTok{(\textquotesingle{}P1\_b \textquotesingle{}, \textquotesingle{}Genero\textquotesingle{})}\StringTok{\textasciigrave{}}\NormalTok{,}
               \StringTok{\textasciigrave{}}\AttributeTok{(\textquotesingle{}P2\_a \textquotesingle{}, \textquotesingle{}Qual sua situação atual de trabalho?\textquotesingle{})}\StringTok{\textasciigrave{}}\NormalTok{)}\SpecialCharTok{\%\textgreater{}\%}
\NormalTok{  dplyr}\SpecialCharTok{::}\FunctionTok{group\_by}\NormalTok{(}\StringTok{\textasciigrave{}}\AttributeTok{(\textquotesingle{}P1\_b \textquotesingle{}, \textquotesingle{}Genero\textquotesingle{})}\StringTok{\textasciigrave{}}\NormalTok{)}\SpecialCharTok{\%\textgreater{}\%}
\NormalTok{  dplyr}\SpecialCharTok{::}\FunctionTok{mutate}\NormalTok{(}\AttributeTok{Percentual =}\NormalTok{ n}\SpecialCharTok{/}\FunctionTok{sum}\NormalTok{(n)}\SpecialCharTok{*}\DecValTok{100}\NormalTok{)}\SpecialCharTok{\%\textgreater{}\%}\NormalTok{.[,}\SpecialCharTok{{-}}\DecValTok{3}\NormalTok{]}
\end{Highlighting}
\end{Shaded}

\begin{verbatim}
## # A tibble: 4 x 3
## # Groups:   ('P1_b ', 'Genero') [2]
##   `('P1_b ', 'Genero')` `('P2_a ', 'Qual sua situação atual de trab~` Percentual
##   <chr>                 <chr>                                              <dbl>
## 1 Feminino              Empreendedor ou Empregado (CNPJ)                    9.63
## 2 Feminino              Empregado (CLT)                                    90.4 
## 3 Masculino             Empreendedor ou Empregado (CNPJ)                   14.4 
## 4 Masculino             Empregado (CLT)                                    85.6
\end{verbatim}

\begin{Shaded}
\begin{Highlighting}[]
\CommentTok{\#Situação por gênero e atuação}
\NormalTok{df}\SpecialCharTok{\%\textgreater{}\%}
\NormalTok{  dplyr}\SpecialCharTok{::}\FunctionTok{count}\NormalTok{(}\StringTok{\textasciigrave{}}\AttributeTok{(\textquotesingle{}P4\_a \textquotesingle{}, \textquotesingle{}Atuacao\textquotesingle{})}\StringTok{\textasciigrave{}}\NormalTok{,}\StringTok{\textasciigrave{}}\AttributeTok{(\textquotesingle{}P1\_b \textquotesingle{}, \textquotesingle{}Genero\textquotesingle{})}\StringTok{\textasciigrave{}}\NormalTok{,}
               \StringTok{\textasciigrave{}}\AttributeTok{(\textquotesingle{}P2\_a \textquotesingle{}, \textquotesingle{}Qual sua situação atual de trabalho?\textquotesingle{})}\StringTok{\textasciigrave{}}\NormalTok{) }\SpecialCharTok{\%\textgreater{}\%}
\NormalTok{  dplyr}\SpecialCharTok{::}\FunctionTok{group\_by}\NormalTok{(}\StringTok{\textasciigrave{}}\AttributeTok{(\textquotesingle{}P1\_b \textquotesingle{}, \textquotesingle{}Genero\textquotesingle{})}\StringTok{\textasciigrave{}}\NormalTok{,}\StringTok{\textasciigrave{}}\AttributeTok{(\textquotesingle{}P4\_a \textquotesingle{}, \textquotesingle{}Atuacao\textquotesingle{})}\StringTok{\textasciigrave{}}\NormalTok{) }\SpecialCharTok{\%\textgreater{}\%}
\NormalTok{  dplyr}\SpecialCharTok{::}\FunctionTok{mutate}\NormalTok{(}\AttributeTok{Prop =}\NormalTok{ n}\SpecialCharTok{/}\FunctionTok{sum}\NormalTok{(n))}\SpecialCharTok{\%\textgreater{}\%}
\NormalTok{  dplyr}\SpecialCharTok{::}\FunctionTok{filter}\NormalTok{(}\StringTok{\textasciigrave{}}\AttributeTok{(\textquotesingle{}P4\_a \textquotesingle{}, \textquotesingle{}Atuacao\textquotesingle{})}\StringTok{\textasciigrave{}}\SpecialCharTok{\%notin\%}\FunctionTok{c}\NormalTok{(}\StringTok{\textquotesingle{}Outra\textquotesingle{}}\NormalTok{, }\StringTok{\textquotesingle{}Gestor\textquotesingle{}}\NormalTok{))}\SpecialCharTok{\%\textgreater{}\%}
\NormalTok{  ggplot2}\SpecialCharTok{::}\FunctionTok{ggplot}\NormalTok{(}
\NormalTok{    ggplot2}\SpecialCharTok{::}\FunctionTok{aes}\NormalTok{(}\AttributeTok{x =} \StringTok{\textasciigrave{}}\AttributeTok{(\textquotesingle{}P1\_b \textquotesingle{}, \textquotesingle{}Genero\textquotesingle{})}\StringTok{\textasciigrave{}}\NormalTok{, }\AttributeTok{y =}\NormalTok{ Prop,}
                 \AttributeTok{fill =} \StringTok{\textasciigrave{}}\AttributeTok{(\textquotesingle{}P2\_a \textquotesingle{}, \textquotesingle{}Qual sua situação atual de trabalho?\textquotesingle{})}\StringTok{\textasciigrave{}}\NormalTok{)) }\SpecialCharTok{+}
\NormalTok{  ggplot2}\SpecialCharTok{::}\FunctionTok{geom\_col}\NormalTok{( }\AttributeTok{color=}\StringTok{"white"}\NormalTok{,}
                     \AttributeTok{position =}\NormalTok{ ggplot2}\SpecialCharTok{::}\FunctionTok{position\_fill}\NormalTok{()) }\SpecialCharTok{+}
\NormalTok{  ggrepel}\SpecialCharTok{::}\FunctionTok{geom\_label\_repel}\NormalTok{(ggplot2}\SpecialCharTok{::}\FunctionTok{aes}\NormalTok{(}
    \AttributeTok{label =}\NormalTok{ scales}\SpecialCharTok{::}\FunctionTok{percent}\NormalTok{(Prop)),}
    \AttributeTok{fontface =} \StringTok{\textquotesingle{}bold\textquotesingle{}}\NormalTok{,}
    \AttributeTok{hjust=}\DecValTok{2}\NormalTok{,}
    \AttributeTok{position =}\NormalTok{ ggplot2}\SpecialCharTok{::}\FunctionTok{position\_stack}\NormalTok{(}\AttributeTok{vjust =}\NormalTok{ .}\DecValTok{5}\NormalTok{),}
    \AttributeTok{size=}\FloatTok{3.5}\NormalTok{) }\SpecialCharTok{+}
\NormalTok{  ggplot2}\SpecialCharTok{::}\FunctionTok{ggtitle}\NormalTok{(}\StringTok{\textquotesingle{}Figura 19: Situação de Trabalho x Atuação x Gênero\textquotesingle{}}\NormalTok{)}\SpecialCharTok{+}
\NormalTok{  ggplot2}\SpecialCharTok{::}\FunctionTok{theme\_void}\NormalTok{()}\SpecialCharTok{+}
\NormalTok{  ggplot2}\SpecialCharTok{::}\FunctionTok{scale\_fill\_brewer}\NormalTok{(}\AttributeTok{type =} \StringTok{"seq"}\NormalTok{, }\AttributeTok{palette =} \StringTok{"Blues"}\NormalTok{)}\SpecialCharTok{+}
\NormalTok{  ggplot2}\SpecialCharTok{::}\FunctionTok{xlab}\NormalTok{(}\StringTok{\textquotesingle{}Gênero\textquotesingle{}}\NormalTok{)}\SpecialCharTok{+}
\NormalTok{  ggplot2}\SpecialCharTok{::}\FunctionTok{theme}\NormalTok{(}\AttributeTok{legend.position=}\StringTok{"bottom"}\NormalTok{,}
                 \AttributeTok{plot.title=}\NormalTok{ggplot2}\SpecialCharTok{::}\FunctionTok{element\_text}\NormalTok{(}\AttributeTok{face=}\StringTok{\textquotesingle{}bold.italic\textquotesingle{}}\NormalTok{,}
                                                  \AttributeTok{hjust =} \FloatTok{0.5}\NormalTok{, }\AttributeTok{size=}\DecValTok{20}\NormalTok{),}
                 \AttributeTok{axis.text.y=}\NormalTok{ggplot2}\SpecialCharTok{::}\FunctionTok{element\_blank}\NormalTok{(),}
                 \AttributeTok{axis.title.y=}\NormalTok{ggplot2}\SpecialCharTok{::}\FunctionTok{element\_blank}\NormalTok{(),}
                 \AttributeTok{axis.title.x=}\NormalTok{ggplot2}\SpecialCharTok{::}\FunctionTok{element\_blank}\NormalTok{(),}
                 \AttributeTok{axis.text.x =}\NormalTok{ggplot2}\SpecialCharTok{::}\FunctionTok{element\_text}\NormalTok{(}\AttributeTok{face=}\StringTok{\textquotesingle{}bold\textquotesingle{}}\NormalTok{, }\AttributeTok{size=}\DecValTok{12}\NormalTok{),}
                 \AttributeTok{legend.title=}\NormalTok{ggplot2}\SpecialCharTok{::}\FunctionTok{element\_blank}\NormalTok{())}\SpecialCharTok{+}
\NormalTok{  ggplot2}\SpecialCharTok{::}\FunctionTok{facet\_grid}\NormalTok{(}\SpecialCharTok{\textasciitilde{}}\StringTok{\textasciigrave{}}\AttributeTok{(\textquotesingle{}P4\_a \textquotesingle{}, \textquotesingle{}Atuacao\textquotesingle{})}\StringTok{\textasciigrave{}}\NormalTok{)}
\end{Highlighting}
\end{Shaded}

\includegraphics{relatorio_files/figure-latex/unnamed-chunk-12-1.pdf}

\begin{Shaded}
\begin{Highlighting}[]
\CommentTok{\#Situação por gênero e nível de cargo}
\NormalTok{df}\SpecialCharTok{\%\textgreater{}\%}
\NormalTok{  dplyr}\SpecialCharTok{::}\FunctionTok{count}\NormalTok{(}\StringTok{\textasciigrave{}}\AttributeTok{(\textquotesingle{}P2\_g \textquotesingle{}, \textquotesingle{}Nivel\textquotesingle{})}\StringTok{\textasciigrave{}}\NormalTok{,}\StringTok{\textasciigrave{}}\AttributeTok{(\textquotesingle{}P1\_b \textquotesingle{}, \textquotesingle{}Genero\textquotesingle{})}\StringTok{\textasciigrave{}}\NormalTok{,}
               \StringTok{\textasciigrave{}}\AttributeTok{(\textquotesingle{}P2\_a \textquotesingle{}, \textquotesingle{}Qual sua situação atual de trabalho?\textquotesingle{})}\StringTok{\textasciigrave{}}\NormalTok{) }\SpecialCharTok{\%\textgreater{}\%}
\NormalTok{  dplyr}\SpecialCharTok{::}\FunctionTok{group\_by}\NormalTok{(}\StringTok{\textasciigrave{}}\AttributeTok{(\textquotesingle{}P1\_b \textquotesingle{}, \textquotesingle{}Genero\textquotesingle{})}\StringTok{\textasciigrave{}}\NormalTok{,}\StringTok{\textasciigrave{}}\AttributeTok{(\textquotesingle{}P2\_g \textquotesingle{}, \textquotesingle{}Nivel\textquotesingle{})}\StringTok{\textasciigrave{}}\NormalTok{) }\SpecialCharTok{\%\textgreater{}\%}
\NormalTok{  dplyr}\SpecialCharTok{::}\FunctionTok{mutate}\NormalTok{(}\AttributeTok{Prop =}\NormalTok{ n}\SpecialCharTok{/}\FunctionTok{sum}\NormalTok{(n))}\SpecialCharTok{\%\textgreater{}\%}
\NormalTok{  dplyr}\SpecialCharTok{::}\FunctionTok{filter}\NormalTok{(}\StringTok{\textasciigrave{}}\AttributeTok{(\textquotesingle{}P2\_g \textquotesingle{}, \textquotesingle{}Nivel\textquotesingle{})}\StringTok{\textasciigrave{}}\SpecialCharTok{!=}\StringTok{\textquotesingle{}Outra\textquotesingle{}}\NormalTok{)}\SpecialCharTok{\%\textgreater{}\%}
\NormalTok{  ggplot2}\SpecialCharTok{::}\FunctionTok{ggplot}\NormalTok{(}
\NormalTok{    ggplot2}\SpecialCharTok{::}\FunctionTok{aes}\NormalTok{(}\AttributeTok{x =} \StringTok{\textasciigrave{}}\AttributeTok{(\textquotesingle{}P1\_b \textquotesingle{}, \textquotesingle{}Genero\textquotesingle{})}\StringTok{\textasciigrave{}}\NormalTok{, }\AttributeTok{y =}\NormalTok{ Prop,}
                 \AttributeTok{fill =} \StringTok{\textasciigrave{}}\AttributeTok{(\textquotesingle{}P2\_a \textquotesingle{}, \textquotesingle{}Qual sua situação atual de trabalho?\textquotesingle{})}\StringTok{\textasciigrave{}}\NormalTok{)) }\SpecialCharTok{+}
\NormalTok{  ggplot2}\SpecialCharTok{::}\FunctionTok{geom\_col}\NormalTok{( }\AttributeTok{color=}\StringTok{"white"}\NormalTok{,}
                     \AttributeTok{position =}\NormalTok{ ggplot2}\SpecialCharTok{::}\FunctionTok{position\_fill}\NormalTok{()) }\SpecialCharTok{+}
\NormalTok{  ggrepel}\SpecialCharTok{::}\FunctionTok{geom\_label\_repel}\NormalTok{(ggplot2}\SpecialCharTok{::}\FunctionTok{aes}\NormalTok{(}
    \AttributeTok{label =}\NormalTok{ scales}\SpecialCharTok{::}\FunctionTok{percent}\NormalTok{(Prop)),}
    \AttributeTok{fontface =} \StringTok{\textquotesingle{}bold\textquotesingle{}}\NormalTok{,}
    \AttributeTok{hjust=}\DecValTok{2}\NormalTok{,}
    \AttributeTok{position =}\NormalTok{ ggplot2}\SpecialCharTok{::}\FunctionTok{position\_stack}\NormalTok{(}\AttributeTok{vjust =}\NormalTok{ .}\DecValTok{5}\NormalTok{),}
    \AttributeTok{size=}\FloatTok{3.5}\NormalTok{) }\SpecialCharTok{+}
\NormalTok{  ggplot2}\SpecialCharTok{::}\FunctionTok{ggtitle}\NormalTok{(}\StringTok{\textquotesingle{}Figura 20: Situação de Trabalho x Nível de Cargo x Gênero\textquotesingle{}}\NormalTok{)}\SpecialCharTok{+}
\NormalTok{  ggplot2}\SpecialCharTok{::}\FunctionTok{theme\_void}\NormalTok{()}\SpecialCharTok{+}
\NormalTok{  ggplot2}\SpecialCharTok{::}\FunctionTok{scale\_fill\_brewer}\NormalTok{(}\AttributeTok{type =} \StringTok{"seq"}\NormalTok{, }\AttributeTok{palette =} \StringTok{"Blues"}\NormalTok{)}\SpecialCharTok{+}
\NormalTok{  ggplot2}\SpecialCharTok{::}\FunctionTok{xlab}\NormalTok{(}\StringTok{\textquotesingle{}Gênero\textquotesingle{}}\NormalTok{)}\SpecialCharTok{+}
\NormalTok{  ggplot2}\SpecialCharTok{::}\FunctionTok{theme}\NormalTok{(}\AttributeTok{legend.position=}\StringTok{"bottom"}\NormalTok{,}
                 \AttributeTok{plot.title=}\NormalTok{ggplot2}\SpecialCharTok{::}\FunctionTok{element\_text}\NormalTok{(}\AttributeTok{face=}\StringTok{\textquotesingle{}bold.italic\textquotesingle{}}\NormalTok{,}
                                                  \AttributeTok{hjust =} \FloatTok{0.5}\NormalTok{, }\AttributeTok{size=}\DecValTok{20}\NormalTok{),}
                 \AttributeTok{axis.text.y=}\NormalTok{ggplot2}\SpecialCharTok{::}\FunctionTok{element\_blank}\NormalTok{(),}
                 \AttributeTok{axis.title.y=}\NormalTok{ggplot2}\SpecialCharTok{::}\FunctionTok{element\_blank}\NormalTok{(),}
                 \AttributeTok{axis.title.x=}\NormalTok{ggplot2}\SpecialCharTok{::}\FunctionTok{element\_blank}\NormalTok{(),}
                 \AttributeTok{axis.text.x =}\NormalTok{ggplot2}\SpecialCharTok{::}\FunctionTok{element\_text}\NormalTok{(}\AttributeTok{face=}\StringTok{\textquotesingle{}bold\textquotesingle{}}\NormalTok{, }\AttributeTok{size=}\DecValTok{12}\NormalTok{),}
                 \AttributeTok{legend.title=}\NormalTok{ggplot2}\SpecialCharTok{::}\FunctionTok{element\_blank}\NormalTok{())}\SpecialCharTok{+}
\NormalTok{  ggplot2}\SpecialCharTok{::}\FunctionTok{facet\_grid}\NormalTok{(}\SpecialCharTok{\textasciitilde{}}\StringTok{\textasciigrave{}}\AttributeTok{(\textquotesingle{}P2\_g \textquotesingle{}, \textquotesingle{}Nivel\textquotesingle{})}\StringTok{\textasciigrave{}}\NormalTok{)}
\end{Highlighting}
\end{Shaded}

\includegraphics{relatorio_files/figure-latex/unnamed-chunk-12-2.pdf}

Mesmo sendo mais escolarizadas no geral, sendo que em nenhum dos níveis
de cargo dá para dizer que os homens são mais escolarizados e o mesmo
vale em relação às funções, incluindo a engenharia de dados, onde, elas
são consideravelmente mais novas e menos experientes que os homens, as
mulheres tendem a receber menos nos cargos mais importantes e em certos
casos mais dificuldade em assumí-los. O que mais parece explicar isso é
a acachapante ``derrota'' das mulheres para os homens em relação a
experiências anteriores em TI, resultado de décadas de exclusão do
gênero feminino nestas áreas. No entanto, é preciso investigar
profundamente se esta maior experiência se traduz em maior produtividade
dentro das empresas ou se é apenas mais uma desculpa para manter os
maiores salários e cargos principais nas mãos dos mesmos de sempre.

\end{document}
